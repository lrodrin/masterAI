\documentclass[a4paper,10pt,twoside]{report}

\usepackage{Packages}

\addto\captionsbritish{\renewcommand{\contentsname}{Tabla de Contenidos}}

%\linenumbers

\newcommand{\shortdoctitle}{Doc title}
\newcommand{\doctitle}{Formalización del problema de la partición del grafo}

\newcommand{\authorone}{Laura Rodríguez Navas \\ rodrigueznavas@posgrado.uimp.es}
\newcommand{\monthYear}{Mayo 2020}

\author{\me}

\hypersetup
{
    pdfauthor   = {\authorone},
    pdftitle    = {\shortdoctitle},
    pdfsubject  = {\doctitle}
}

\begin{document}

\begin{titlepage}
\begin{center}
\includegraphics[height=2.1cm]{Figures/uimp-logo.png} \\
\LARGE
Resolución de problemas con metaheurísticos \\
\Large
Informe \\
\large

\vspace*{10cm}

\setlength{\TPHorizModule}{1mm}
\setlength{\TPVertModule}{\TPHorizModule}

\newlength{\backupparindent}
\setlength{\backupparindent}{\parindent}
\setlength{\parindent}{0mm}

\begin{textblock}{155}(32,89)
    \vspace*{1mm}
    \huge
    \textbf{\doctitle \\}
    \Large
    \vspace*{5mm}
    % \textit{\docsubtitle} \\
    \vspace*{5mm}
    \Large
     \begin{tabular}{c c c}
            \authorone
    \end{tabular} \\
\end{textblock}

\vfill
%\docdate \\
\large
\monthYear \\

\setlength{\parindent}{\backupparindent}
\end{center}
\end{titlepage} 

\normalsize

\chapter*{Resumen}\label{chapter:Resumen}
\setcounter{page}{0}
\pagenumbering{arabic}
\addcontentsline{toc}{chapter}{Resumen}
In mathematics, a graph partition is the reduction of a graph to a smaller graph by partitioning its nodes into mutually exclusive groups. It is commonly used in scientific computing, VLSI circuit design, and task scheduling in multiprocessor computers, among others. Actually, finding partitions that simplifies graph analysis is a hard problem and the graph partitioning problem has gained importance due to its application for clustering and detection of cliques in social, pathological and biological networks. 

Since graph partitioning is a hard problem, the practical solutions are based on heuristics, because typically graph partition problem fall under the category of NP-hard problems. For tackling this problem, I propose three algorithms based on heuristics, including local and global strategies. The two first algorithms presented are Kernighan–Lin and Fiduccia-Mattheyses, effective 2-way cuts by local search strategies. The third is the Spectral Bisection algorithm. Finally, I demonstrate with some examples my approach in an evaluation of these algorithms using a dataset.




\tableofcontents

\chapter{Introducción a la Partición de Grafos}\label{chapter:Introducción}
\section{Introducción}
En los últimos años el problema de la partición de grafos ha sido ampliamente estudiado. El problema se deriva de situaciones del mundo real por lo que tiene aplicación en varias ramas de la ingeniería y de la investigación. La partición de grafos es una disciplina ubicada entre las ciencias computacionales y la matemática aplicada. El problema es un problema importante que tiene aplicaciones extensas en muchas áreas ya que se busca minimizar los costes o maximizar los beneficios a través de la correcta distribución de recursos. 

La primera aplicación del problema fue en la partición de componentes de un circuito en un conjunto de tarjetas que juntas realizaban tareas en un dispositivo electrónico. Las tarjetas tenían un tamaño limitado, de tal manera que el dispositivo no llegara a ser muy grande, y el número de elementos de cada tarjeta estaba restringido. Si el circuito era demasiado grande podía ser dividido en varias tarjetas las cuales estaban interconectadas, sin embargo, el coste de interconexión era muy elevado por el que el número de interconexiones debía ser minimizado.

La aplicación descrita anteriormente fue presentada en \cite{KernighanLin}, debido a Kernighan y Lin (1970), en la cuan se define un algoritmo eficiente para encontrar buenas soluciones. En la aplicación, el circuito es asociado a un grafo y las tarjetas como subconjuntos de una partición. Los nodos del grafo son representados por los componentes electrónicos y las aristas forman las interconexiones entre los componentes y las tarjetas.

\section{Descripción el problema}
El problema de partición de grafos puede formularse como un problema de programación entera. Los problemas de programación entera generalmente están relacionados directamente a problemas combinatorios, esto genera que al momento de resolver los problemas de programación entera se encuentren restricciones dado el coste computacional de resolverlos; por ese motivo se han desarrollado algoritmos que buscan soluciones de una forma más directa, la restricción de los mismos es que no se puede garantizar que la solución encontrada sea la óptima. 

Este problema ha sido demostrado como un problema NP-completo (\url{https://es.wikipedia.org/wiki/NP-hard}), lo que implica que las soluciones para él no pueden ser encontradas en tiempos razonables.

En el presente trabajo se pretende estudiar el algoritmo descrito en \cite{KernighanLin}, \cite{FiducciaMattheyses} i 

\chapter{Algoritmos basados en metaheurísticas}\label{chapter:Algoritmos}
Debido a la \textit{"intratabilidad"} del problema de la partición de un grafo, no existe un algoritmo concreto que permita obtener en tiempo polinómico una solución óptima a la partición de cualquier grafo. Es por ello que, los algoritmos codificados para este trabajo se basan en la metaheurística, con el objetivo de obtener soluciones de buena calidad en tiempos computacionales aceptables, a pesar de que los algoritmos metaheurísticos no garantizan que se vaya a obtener una solución óptima al problema, incluso podría suceder que no se obtenga ninguna solución. 

En el siguiente apartado se analizan los algoritmos Kernighan-Lin\cite{KernighanLin} (ver apartado \ref{Kernighan-Lin}), Specrtal Bisection (ver apartado \ref{Spectral-Bisection}) y Multilevel Spectral Bisection (ver apartado \ref{Multilevel-Spectral-Bisection}), tres algoritmos diseñados específicamente para la resolución del problema de la partición de grafos. Cualquiera de estos algoritmos proporciona una solución factible al problema, pudiendo ser esta óptima o no. Además de describir cada uno de los algoritmos detalladamente, las principales propiedades de cada uno de ellos, también se describirán algunos ejemplos.

\section{Kernighan-Lin}\label{Kernighan-Lin}

El algoritmo Kernighan-Lin\cite{KernighanLin}, a menudo abreviado como K-L, es uno de los primeros algoritmos de partición de grafos y fue desarrollado originalmente para optimizar la colocación de circuitos electrónicos en tarjetas de circuito impreso para minimizar el número de conexiones entre las tarjetas.

El algoritmo K-L no crea particiones, sino que las mejora iterativamente. La idea original era tomar una partición aleatoria y aplicarle Kernighan-Lin\cite{KernighanLin}. Esto se repetiría varias veces y se elegiría el mejor resultado. Mientras que para grafos pequeños esto ofrece resultados razonables, es bastante ineficiente para tamaños de problemas más grandes.

Decimos que es un algoritmo:

\begin{itemize}
	\setlength{\parskip}{0pt}
	\setlength{\itemsep}{0pt plus 1pt}
	\item \textbf{Iterativo}. Esto significa que el grafo inicialmente ya está particionado, pero la aplicación del algoritmo intentará mejorar u optimizar la partición inicial. 
	\item \textbf{Voraz}. Esto significa que el algoritmo hará cambios si hay un beneficio inmediato sin considerar otras formas posibles de obtener una solución óptima.
	\item \textbf{Determinista} porque se obtendrá el mismo resultado cada vez que se aplique el algoritmo. 
\end{itemize}

%Una aplicación importante del algoritmo es en los circuitos VLSI\cite{KernighanLin}\cite{Ravikumar}. Concretamente se usa para encontrar un número mínimo de conexiones entre particiones para mejorar la velocidad o disminuir el consumo de energía.

Hoy en día, el algoritmo se usa para mejorar las particiones encontradas por otros algoritmos. Como veremos, K-L tiene una visión algo \textit{"local"} del problema, tratando de mejorar las particiones mediante el intercambio de nodos vecinos. Por lo tanto, complementa muy bien los algoritmos que tienen una visión más \textit{"global"} del problema, pero tienden a ignorar las características locales. Un ejemplo de este tipo de algoritmos seria la partición espectral (ver sección \ref{Spectral-Bisection}).

C. Fiduccia y R. Mattheyses realizaron importantes avances prácticos en \cite{FiducciaMattheyses} quienes mejoraron el algoritmo de tal manera que una iteración se ejecuta en $O({n}^2)$ en lugar de $O({n}^2 \, log \, n)$.

\subsection{Descripción}

Ventajas:

\begin{itemize}
	\item El algoritmo es robusto.
\end{itemize}

Desventajas:

\begin{itemize}
	\item Los resultados son aleatorios porque el algoritmo comienza con una partición aleatoria.
	\item Computacionalmente es un algoritmo lento.
	\item Solo se crean dos particiones del mismo tamaño.
	\item Las particiones tienen que tener el mismo tamaño para que el algoritmo no intente encontrar tamaños de partición óptimos que ya existan.
	\item No resuelve muy bien los problemas con aristas ponderadas.
	\item La solución dependerá en gran medida del primer intercambio.
\end{itemize}

\subsection{Ejemplo}

\section{Spectral Bisection}\label{Spectral-Bisection}

\subsection{Descripción}

\subsection{Ejemplo}

\section{Multilevel Spectral Bisection}\label{Multilevel-Spectral-Bisection}

El particionamiento de grafos por multilevel es un método moderno que reduce el tamaño de las particiones del grafo con la combinación de los vértices y las aristas sobre varios niveles, creando particiones del grafo cada vez más pequeñas y extensas con muchas variaciones y combinaciones de diferentes métodos.

METIS es un algoritmo de partición que se enfoca en minimizar el número de bordes de partición cruzados y distribuir la carga de trabajo de manera uniforme entre las particiones. Con METIS, los gráficos se dividen en tres fases. La primera fase es la fase de engrosamiento, la segunda la fase de partición y la tercera y última fase la fase de no engrosamiento (Figura X). La fase de particionamiento contiene particiones bi-particionadas y K-way. A diferencia de la partición K-way, la bi-partición se realiza de forma recursiva. Las subseccion a continuación contienen una explicación más extensa de estas diversas fases.

\subsection{Descripción}

\subsection{Ejemplo}



%\chapter{Experimental set-up}\label{chapter:Experimental set-up}
%\input{Chapters/Chapter3_Experimental_set-up.tex}

%\chapter{Results and discussion}\label{chapter:Results and discussion}
%\input{Chapters/Chapter4_Results_and_discussion.tex}

\chapter{Conclusiones}\label{chapter:Conclusiones}
METIS es el algoritmo más eficiente en la partición de grafos que se han codificado.

\bibliographystyle{unsrt}
\bibliography{References}

%\newpage

%\appendix
%\addcontentsline{toc}{chapter}{Appendix}
%\chapter{Appendix}\label{chapter:Appendix A}
%\input{Appendices/Appendix_A.tex}

\end{document}