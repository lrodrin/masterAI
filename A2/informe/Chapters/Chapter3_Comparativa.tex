En esta sección, veremos una comparativa entre los algoritmos que hemos descrito en las secciones previas, tras ser aplicados sobre diferentes grafos aleatorios, también conocidos como grafos de Erdös y Renyi\cite{ErdosRenyi} o grafos binomiales. Pero antes de ver la comparativa, en primer lugar, definimos el concepto de grafo aleatorio.

Se dice que un grafo es aleatorio si la presencia y colocación de sus aristas siguen una distribución aleatoria. Por tanto, un grafo aleatorio no puede diseñarse bajo ningún criterio concreto. Los grafos aleatorios también son conocidos como grafos de Erdös y Renyi porqué utilizan el modelo de Erdös y Renyi. El nombre de este modelo proviene de los matemáticos Paul Erdős y Alfréd Rényi, quienes lo presentaron por primera en 1959.

El modelo de Erdös y Renyi (a veces abreviado como modelo ER), es el modelo que se ha empleado para la generación de los grafos aleatorios de prueba que se usan en la comparativa de las codificaciones. Para ello, los grafos se han construido mediante la conexión de los nodos al azar. Cada arista se incluye en el grafo con una probabilidad de \textit{p} independiente de las otras aristas del grafo. De manera equivalente, todos los grafos con \textit{n} nodos y \textit{m} aristas tienen la misma probabilidad. 

La probabilidad $p$ se define:

\begin{center}
	$p^m(1 - p)^{{n \choose 2} - m}$
\end{center}

En la implementación de los grafos aleatorios se ha utilizado la librería de Python:  \href{https://networkx.github.io/documentation/latest/reference/generated/networkx.generators.random\_graphs.erdos\_renyi\_graph.html?highlight=erdos\%20renyi#networkx.generators.random\_graphs.erdos\_renyi\_graph}{NetworkX}. Concretamente el método que se ha utilizado nos devuelve un grafo no dirigido, donde se elige cada una de las aristas posibles con probabilidad \textit{p} = 0.7, según el número de nodos de entrada \textit{n}. En particular, el caso \textit{p} = 0.7 se corresponde con el caso en el que los \textit{n} vértices se eligen con mayor probabilidad (\textit{p} $>$ 0.5). Los pesos de las aristas también se han generado aleatoriamente dentro de un intervalo de 1 a 20.

Podemos observar la codificación acabada de describir, que se ejecuta en tiempo $O$($n^2$).

\lstset{language=Python}    
\begin{lstlisting}[frame=single]  
G = nx.erdos_renyi_graph(n, 0.7)

for u, v in G.edges():
  if u != v:
    G[u][v]['label'] = random.randrange(1, 20)
\end{lstlisting}

Y en la siguiente tabla, se presenta la comparación entre los algoritmos codificados sobre los grafos aleatorios generados en este informe, en términos de tiempo computacional. Se muestra el tiempo (en segundos) en completar las ejecuciones de los algoritmos sobre los grafos aleatorios para distintos números de vértices \textit{n}. Para ello se ha utilizado un MacBook Pro, con un procesador de cuatro núcleos de 2.8 GHz y con 16 GB de memoria RAM. Es de esperar que, a mayor número de vértices, los algoritmos hayan tenido mayor tiempo de ejecución.

\renewcommand{\tablename}{Tabla}
\begin{table}[h]	
	\begin{center}
		\begin{tabular}{|c|c|c|c|}
			\hline
			\textit{n} & 307 & 552 & 861 \\
			\hline
			Kernighan-Lin & 0.0117 & 0.0214 & 0.0451\\
			\hline
			Spectral Bisection & 0.0021 & 0.0031 & 0.0060 \\
			\hline
			MSB & 0.0025 & 0.0031 & 0.0037 \\ 
			\hline
		\end{tabular}
		\vspace{3mm}
		\caption{Tabla comparativa de los algoritmos.}
	\end{center}
\end{table}

Después de la comparación, podemos ver como los algoritmos Kernighan-Lin, Spectral Bisection y Multilevel Spectral Bisection (MSB) tienen unos tiempos de ejecución similares cuando el número de vértices \textit{n} es más pequeño. Fenómeno que cambia cuando casi se duplican el número de vértices a 552. 

\newpage
Además, si nos fijamos en el algoritmo Kernighan-Lin, cuando el número de vértices casi se duplica a 552, el tiempo de ejecución del algoritmo también se duplica. Parece que el algoritmo va aumentando el tiempo de ejecución a medida que aumenta el número de vértices de manera proporcionada.

En cambio, el algoritmo Spectral Bisection obtiene unos resultados muy parecidos con el algoritmo Multilevel Spectral Bisection (MSB), pero cuando el número de vértices empieza a ser muy elevado, la eficiencia de este algoritmo baja drásticamente. Que no es el caso del algoritmo Multilevel Spectral Bisection (MSB).

Como conclusión general de esta breve comparación entre los algoritmos podemos decir que el algoritmo Multilevel Spectral Bisection (MSB) es el más eficiente, con tiempos de ejecución más pequeños, y el algoritmo Kernighan-Lin es el algoritmo menos eficiente, con tiempos de ejecución más grandes. También podemos observar que los tiempos de ejecución no son muy grandes, ya que solo se ha medido el tiempo de ejecución de los algoritmos y además el número de vértices totales no es muy grande. Aun así, nos ha permitido ver las diferencias en tiempo de ejecución de las tres codificaciones que se han implementado y descrito en este informe.

La elección del tiempo de ejecución frente a la calidad de las particiones para la comparación de los algoritmos ha dependido del hecho que el problema de partición de grafos sea un problema NP-completo\cite{NPCompleteness}. Recordemos que esto quiere decir que los tres algoritmos puede que no nos ofrezcan una solución óptima en cada ejecución, pero sí que nos dan una buena solución, al menos la mayor parte del tiempo. Por esto, como es difícil comparar la calidad de las particiones entre los diferentes algoritmos, se elijó la medida del tiempo de ejecución de cada algoritmo. Por otro lado, bajo el contexto de matriz-vector de los algoritmos Spectral Bisection y Multilevel Spectral Bisection (MSB), también solo nos interesaba el tiempo de ejecución total. 

Concretamente en este caso, podría ser que, si solo utilizamos una vez una determinada matriz, el algoritmo que entrega solo particiones de calidad media en general podría ser más rápido que el algoritmo más lento con particiones de mejor calidad. Pero si usamos la misma matriz (o diferentes matrices con el mismo grafo) a menudo, el algoritmo más lento podría ser preferible. De hecho, hay incluso más factores a considerar. Hasta ahora hemos querido que las diferentes particiones tuvieran del mismo tamaño. Pero como hemos visto en la sección 2.2, podría ser una ventaja aceptar particiones de un tamaño ligeramente diferente para lograr una mejor solución. Todo esto debería demostrar que no existe un mejor algoritmo único para todas las situaciones y que los diferentes algoritmos descritos en las anteriores secciones pueden tener diferentes aplicaciones.

Aunque en general, los algoritmos de particionamiento complejos tardan más en calcular las particiones, ya que crean particiones que a menudo son más equilibradas y/o minimizan con mayor número las conexiones entre particiones, podemos considerar que algoritmos de partición menos complejos sean más eficientes, porqué el objetivo principal de los algoritmos de partición siempre es reducir el tiempo de cálculo en una ejecución. En consecuencia, en algunos casos un algoritmo menos complejo sería preferible a uno más complejo. También nos hemos encontrado con que la eficiencia cambia con el número de particiones y que los algoritmos producen resultados diferentes dependiendo de los atributos del grafo de entrada inicial.