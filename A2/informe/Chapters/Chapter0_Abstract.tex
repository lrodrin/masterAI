In mathematics, a graph partition is the reduction of a graph to a smaller graph by partitioning its nodes into mutually exclusive groups. It is commonly used in scientific computing, VLSI circuit design, and task scheduling in multiprocessor computers, among others. Actually, finding partitions that simplifies graph analysis is a hard problem and the graph partitioning problem has gained importance due to its application for clustering and detection of cliques in social, pathological and biological networks. 

Since graph partitioning is a hard problem, the practical solutions are based on heuristics, because typically graph partition problem fall under the category of NP-hard problems. For tackling this problem, I propose three algorithms based on heuristics, including local and global strategies. The two first algorithms presented are Kernighan–Lin and Fiduccia-Mattheyses, effective 2-way cuts by local search strategies. The third is the Spectral Bisection algorithm. Finally, I demonstrate with some examples my approach in an evaluation of these algorithms using a dataset.


