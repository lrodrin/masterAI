Hemos utilizado el mismo grafo aleatorio para los tres ejemplos descritos de las codificaciones de los algoritmos.

El algoritmo Multilevel Spectral Bisection es el más eficiente que se ha codificado, y el algoritmo Kernighan-Lin es el menos eficiente.

Para la comparativa solo se ha tenido en cuenta el tiempo de ejecución de los algoritmos. No se ha añadido el tiempo de ejecución empleado en la creación de los grafos, antes y después de la partición.

Cada vez que ejecutamos un algoritmo las soluciones cambian. Así que, por eso, los hemos comparado por tiempo de ejecución. Ya que, al ser algoritmos metaheurísticos, todas las soluciones pueden ser óptimas y válidas. Eso quiere decir, que, aunque un algoritmo sea más eficiente, no obtendrá la solución óptima.

Como hemos visto, a menudo existe una relación entre el tiempo de ejecución y la calidad de la solución. Algunos algoritmos funcionan muy rápido pero solo encuentran una solución de medio calidad mientras que otros tardan mucho tiempo pero ofrecen soluciones excelentes e incluso otros se puede sintonizar entre ambos extremos.

La elección del tiempo frente a la calidad.

Las particiones pequeñas en general podrían ser más rápidas que un algoritmo más lento con mejor calidad
particiones Pero si usamos la misma matriz (o diferentes matrices con el mismo gráfico)
a menudo, el algoritmo más lento podría ser preferible. tamaño de corte
Todo esto debería demostrar que no existe un mejor algoritmo único para todas las situaciones.
y que los diferentes algoritmos descritos en los siguientes capítulos tienen todos sus
aplicaciones.