Dentro del amplio campo de la Teoría de Grafos, en este trabajo se considera el problema de la partición del grafo: dado un grafo no dirigido cuyo número de vértices es par y cuyas aristas tienen asociado un peso, se trata de encontrar la partición del conjunto de vértices en dos subconjuntos de tamaño similar de manera que se minimice la suma de los pesos de aquellas aristas que unen vértices que están en diferentes conjuntos

En el capítulo inicial se presenta una breve introducción al problema y algunas nociones básicas. En el segundo capítulo, se proponen tres codificaciones para este problema, señalando sus aspectos más relevantes (positivos y negativos), además de ejemplos. Concretamente, las codificaciones son implementaciones en Python 3 de los algoritmos basados en técnicas metaheurísticas: Kernighan-Lin, Spectral Bisection y Multilevel Spectral Bisection. En el tercer capítulo, se concluye con una comparativa entre las diferentes codificaciones. Finalmente, en el último capítulo se presentan algunas conclusiones.

Las codificaciones se encuentran en el repositorio:

\begin{center}
	\url{https://github.com/lrodrin/masterAI/tree/master/A2}\label{GitHub}
\end{center}
