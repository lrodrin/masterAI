Dentro del amplio campo de la Teoría de Grafos, en este informe se considera el problema de la partición del grafo: dado un grafo no dirigido cuyo número de vértices es par y cuyas aristas tienen asociado un peso, se trata de encontrar la partición del conjunto de vértices en dos subconjuntos de tamaño similar de manera que se minimice la suma de los pesos de aquellas aristas que unen vértices que están en diferentes conjuntos.

En el capítulo inicial se presenta una breve introducción y descripción del problema. En el segundo capítulo, se describen tres codificaciones para este problema, señalando sus ventajas y desventajas. Concretamente, se describen estas codificaciones que son implementaciones en Python de los algoritmos basados en técnicas metaheurísticas: Kernighan-Lin, Spectral Bisection y Multilevel Spectral Bisection. En el tercer capítulo, se hace una comparativa entre las diferentes codificaciones. Finalmente, en el último capítulo se presentan algunas conclusiones.

Las codificaciones que se describen en este informe se encuentran en el repositorio:

\begin{center}
	\url{https://github.com/lrodrin/masterAI/tree/master/A2}\label{GitHub}
\end{center}
