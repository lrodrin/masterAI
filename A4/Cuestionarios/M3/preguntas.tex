\documentclass[11pt]{exam}
\usepackage[utf8]{inputenc}
\usepackage{hyperref}

\title{Evaluación del Módulo 3}
\author{Laura Rodríguez Navas \\ rodrigueznavas@posgrado.uimp.es}
\date{Febrero 2020}

\pagestyle{plain}

\begin{document}

\maketitle

\begin{questions}
	
% Pregunta 1
{\bf \question Describa los distintos tipos de recursos lingüísticos que puede usar un sistema de PLN, así como indique al menos dos ejemplos de cada tipo de recurso. Se valoraran positivamente que los ejemplos se refieran a recursos para el procesamiento de Español (2 puntos).}

Los tipos de recursos lingüísticos que puede usar un sistema de PLN son: Lexicones y Diccionarios, Ontologías y Corpus.

Los {\bf Lexicones} son series ordenadas de palabras de una lengua, una persona, una región, una materia o una época determinadas. Y podemos clasificar-los como generales o especializados.

Los lexicones generales son repositorios de palabras. Pueden ser tan simples como una lista de palabras o pueden ser tan complejos como una base de datos terminológica. También incorporan información o conocimiento sobre las palabras; como información fonológica, morfológica, sintáctica, semántica y pragmática.

De lexicones especializados hay de muchos tipos, dentro de distintas categorías, como por ejemplo:

\begin{itemize}
	\item Diccionarios de locuciones.
	\item Bases de datos léxicas.
	\item Gazetteers.
	\item Bases de datos terminológicas.
	\item Listas de nombres propios.
	\item Listas de siglas o jergas.
	\item Detectores de fechas, números, fórmulas, etc.
\end{itemize}

En realidad, los lexicones especializados están orientados a ciertas aplicaciones o a ciertos dominios concretos. Por ejemplo, en las aplicaciones de análisis de sentimientos se pueden utilizar listas de términos polares; y en las aplicaciones de recuperación de información geográfica se pueden utilizar lista de topónimos e localizaciones. Para los dominios, por ejemplo, de tipo biomédico se pueden usar listas de términos médicos o ontologías médicas; y para dominios de tipo turístico se pueden usar listas de monumentos, ciudades, etc. o también listas de topónimos, como en el caso de las aplicaciones.

Un Diccionario

Una Ontología

Corpus

% Pregunta 2
{\bf \question Explique las aplicaciones de PLN que pueden estar interesadas en el uso de un Gazetter (2 puntos).}

% Pregunta 3
{\bf \question Describa las diferencias entre un reconocedor de entidades y un clasificador de entidades (2 puntos).}

% Pregunta 4
{\bf \question Indique al menos 5 ejemplos de entidades ambiguas (2 puntos).}

% Pregunta 5
{\bf \question Razone si el uso de un reconocedor de entidades sería beneficioso para el Análisis de Opiniones. El Análisis de Opiniones es la tarea que se encarga de la clasificación automática de opiniones (2 puntos).}

\end{questions}

\end{document}