\documentclass[11pt]{exam}
\usepackage[utf8]{inputenc}
\usepackage{hyperref}

\title{Evaluación del Módulo 3}
\author{Laura Rodríguez Navas \\ rodrigueznavas@posgrado.uimp.es}
\date{Febrero 2020}

\pagestyle{plain}

\begin{document}

\maketitle

\begin{questions}
	
% Pregunta 1
{\bf \question Describa los distintos tipos de recursos lingüísticos que puede usar un sistema de PLN, así como indique al menos dos ejemplos de cada tipo de recurso. Se valoraran positivamente que los ejemplos se refieran a recursos para el procesamiento de  Español (2 puntos).}

Los tipos de recursos lingüísticos que puede usar un sistema de PLN son: Lexicones y Diccionarios, Ontologías y Corpus.

Los Lexicones son series ordenadas de palabras de una lengua, una persona, una región, una materia o una época determinadas. 

Podemos clasificar-los como generales o especializados.

\begin{itemize}
	\item Los lexicones generales son repositorios de palabras. Pueden ser tan simples como una lista de palabras o pueden ser tan complejos como una base de datos terminológica. También incorporan información o conocimiento sobre las palabras; como información fonológica, morfológica, sintáctica, semántica y pragmática.
	\item Los lexicones especializados
\end{itemize}

Un Diccionario

Una Ontología

Corpus

% Pregunta 2
{\bf \question Explique las aplicaciones de PLN que pueden estar interesadas en el uso de un Gazetter (2 puntos).}

% Pregunta 3
{\bf \question Describa las diferencias entre un reconocedor de entidades y un clasificador de entidades (2 puntos).}

% Pregunta 4
{\bf \question Indique al menos 5 ejemplos de entidades ambiguas (2 puntos).}

% Pregunta 5
{\bf \question Razone si el uso de un reconocedor de entidades sería beneficioso para el Análisis de Opiniones. El Análisis de Opiniones es la tarea que se encarga de la clasificación automática de opiniones (2 puntos).}

\end{questions}

\end{document}