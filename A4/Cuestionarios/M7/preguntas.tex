\documentclass[11pt]{exam}
\usepackage[utf8]{inputenc}
\usepackage{hyperref}

\title{Evaluación del Módulo 7}
\author{Laura Rodríguez Navas \\ rodrigueznavas@posgrado.uimp.es}
\date{Marzo 2020}

\pagestyle{plain}

\begin{document}
	
\maketitle

\begin{questions}
	
% Pregunta 1
{\bf \question ¿Cuál es el objetivo de la tarea de Generación de Resúmenes? (1 puntos).}

El objetivo de la tarea de Generación de Resúmenes es el de localizar las partes relevantes de un texto, de un documento o de una colección de documentos, de acuerdo con las necesidades de los usuarios, y crear automáticamente un resumen de ello. Este resumen será el resultado de reducir a términos breves y precisos, o considerar tan solo y repetir abreviadamente lo esencial de un asunto o materia.

% Pregunta 2
{\bf \question ¿En qué se diferencia la Generación de Resúmenes de la Extracción de Información? (2 puntos).}

% Pregunta 3
{\bf \question ¿Cuántas categorías de técnicas de Generación de Resúmenes existen? Defínalas (2 puntos).}

Existen dos categorías de técnicas de Generación de Resúmenes automáticas: las técnicas de extracción y las técnicas de abstracción.

Las técnicas de extracción usan la extracción como proceso para identificar la información importante en los textos y realizan un análisis superficial de estos. Los resúmenes extraídos con esta técnica actúan sobre uno o varios documentos, vistos como una colección de oraciones; y de estas oraciones extraen y presentan aquéllas consideradas más relevantes o que responden a unos determinados criterios. El resumen resultante es un subconjunto de las oraciones del texto original.

Las técnicas de abstracción realizan una representación semántica del texto original. El resumen resultante de la representación semántica no consiste en determinadas oraciones entresacadas del texto original, sino en un documento de nueva redacción generado automáticamente a partir del tratamiento de la información contenida del texto original.

% Pregunta 4
{\bf \question Atendiendo a la lección ¿cómo se puede evaluar un sistema de Generación de Resúmenes? (2 puntos).}

Actualmente se viene a diferenciar dos métodos para la evaluación de Generación de Resúmenes automáticos, hablamos de métodos intrínsecos y métodos extrínsecos.


Los métodos de evaluación intrínsecos evalúan directamente del resumen de forma individual, y valoran:

\begin{itemize}
	\item Valoran la calidad del resumen en legibilidad, comprensión, acrónimos, anáforas, integridad de la estructura, gramaticalidad, estilo impersonal.
	\item Valoran la informatividad. La información que contiene respecto a un resumen ideal y con respecto al texto original.
\end{itemize}

Los métodos de evaluación extrínsecos evalúan el uso del resumen en relación con otras tareas:

\begin{itemize}
	\item Encontrar documentos relevantes en una colección.
	\item Decisión tomada leyendo el resumen o el texto original.
	\item Sistemas de recuperación de información.
	\item Contenidos páginas Web (buscadores).
\end{itemize}

% Pregunta 5
{\bf \question Razone cuál es la diferencia entre la tarea Generación de Resúmenes y la tarea Simplificación de Textos. Lectura recomendada: \href{https://pdfs.semanticscholar.org/8a30/f16ed3a83734c4ec087191401a8f241aa3a9.pdf}{A survey of research on text simplification} (3 puntos).}

\end{questions}
	
\end{document}
