\documentclass[11pt]{exam}
\usepackage[utf8]{inputenc}
\usepackage{hyperref}

\title{Evaluación del Módulo 7}
\author{Laura Rodríguez Navas \\ rodrigueznavas@posgrado.uimp.es}
\date{Marzo 2020}

\pagestyle{plain}

\begin{document}
	
\maketitle

\begin{questions}
	
% Pregunta 1
{\bf \question ¿Cuál es el objetivo de la tarea de Generación de Resúmenes? (1 puntos).}

El objetivo de la tarea de Generación de Resúmenes es el de localizar las partes relevantes de un texto, de un documento o de una colección de documentos, de acuerdo con las necesidades de los usuarios, y crear automáticamente un resumen de ello. Este resumen será el resultado de reducir a términos breves y precisos, o considerar tan solo y repetir abreviadamente lo esencial de un asunto o materia.

% Pregunta 2
{\bf \question ¿En qué se diferencia la Generación de Resúmenes de la Extracción de Información? (2 puntos).}

En la generación de resúmenes, una definición a priori de los criterios de interés que se generan, no siempre se define. En cambio, en la extracción de información, la de definición a priori de criterios de interés que se extraen siempre se define a priori.

% Pregunta 3
{\bf \question ¿Cuántas categorías de técnicas de Generación de Resúmenes existen? Defínalas (2 puntos).}

Existen dos categorías de técnicas de Generación de Resúmenes automáticas: las técnicas de extracción y las técnicas de abstracción.

Las técnicas de extracción usan la extracción como proceso para identificar la información importante en los textos y realizan un análisis superficial de estos. Los resúmenes extraídos con esta técnica actúan sobre uno o varios documentos, vistos como una colección de oraciones; y de estas oraciones extraen y presentan aquéllas consideradas más relevantes o que responden a unos determinados criterios. El resumen resultante es un subconjunto de las oraciones del texto original.

Las técnicas de abstracción realizan una representación semántica del texto original. El resumen resultante de la representación semántica no consiste en determinadas oraciones entresacadas del texto original, sino en un documento de nueva redacción generado automáticamente a partir del tratamiento de la información contenida del texto original.

% Pregunta 4
{\bf \question Atendiendo a la lección ¿cómo se puede evaluar un sistema de Generación de Resúmenes? (2 puntos).}

Actualmente se viene a diferenciar dos métodos para la evaluación de Generación de Resúmenes automáticos, hablamos de métodos intrínsecos y métodos extrínsecos.

\newpage

Los métodos de evaluación intrínsecos evalúan directamente del resumen de forma individual, y valoran:

\begin{itemize}
	\item Valoran la calidad del resumen en legibilidad, comprensión, acrónimos, anáforas, integridad de la estructura, gramaticalidad, estilo impersonal.
	\item Valoran la informatividad. La información que contiene respecto a un resumen ideal y con respecto al texto original.
\end{itemize}

Los métodos de evaluación extrínsecos evalúan el uso del resumen en relación con otras tareas:

\begin{itemize}
	\item Encontrar documentos relevantes en una colección.
	\item Decisión tomada leyendo el resumen o el texto original.
	\item Sistemas de recuperación de información.
	\item Contenidos páginas Web (buscadores).
\end{itemize}

% Pregunta 5
{\bf \question Razone cuál es la diferencia entre la tarea Generación de Resúmenes y la tarea Simplificación de Textos. Lectura recomendada: \href{https://pdfs.semanticscholar.org/8a30/f16ed3a83734c4ec087191401a8f241aa3a9.pdf}{A survey of research on text simplification} (3 puntos).}

La simplificación de los textos es el proceso de reducir la complejidad lingüística de un texto, mientras se conserva la información y el significado originales. En términos más generales, la simplificación del texto abarca otras operaciones; por ejemplo, la generación de resúmenes para reducir la longitud de los textos al omitir información periférica o inapropiada. Así, la generación de resúmenes utiliza como metodología la simplificación de textos.

La simplificación de los textos se aplica a tareas de generación de resúmenes de varios documentos, donde se deben generar resúmenes cortos (a menudo de 100 palabras) a partir de un conjunto de oraciones relacionadas. Otros generadores de resúmenes usan la simplificación sintáctica, a menudo como un medio para acortar la oración, extracción previa o posterior a la oración. 

El objetivo principal de la simplificación de los textos es hacer que la información sea más accesible para un gran número de personas con alfabetización reducida u otros problemas de comprensión como la dislexia. Existe evidencia de estudios que utilizan la simplificación de los textos donde la comprensión de lectura puede mejorarse al sustituir palabras difíciles, dividir oraciones largas, hacer explícitas las relaciones de discurso, evitar cláusulas adverbiales prepuestas y presentar información en un orden de causa y efecto. 

\end{questions}
	
\end{document}
