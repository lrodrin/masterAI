\documentclass[11pt]{exam}
\usepackage[utf8]{inputenc}
\usepackage{hyperref}

\title{Evaluación del Módulo 5.2}
\author{Laura Rodríguez Navas \\ rodrigueznavas@posgrado.uimp.es}
\date{Marzo 2020}

\pagestyle{plain}

\begin{document}
	
\maketitle

\begin{questions}

% Pregunta 1
{\bf \question Defina la tarea de Recuperación de Información (2 puntos).}

La Recuperación de Información es la tarea que trata de satisfacer las necesidades de información de los usuarios, es decir, ante una consulta de un usuario, un sistema de recuperación de información busca en una colección de documentos aquel documento o fragmento de texto que resuelve la consulta planteada por el usuario. Ejemplos de sistemas de recuperación de información son los buscadores que se encuentran disponibles en Internet, como es el caso de Google o Bing.

En esta lección aprenderás:

Modelos de recuperación de información: Modelo Booleano y Modelo Espacio Vectorial.
Medidas de evaluación de un sistema de recuperación de la información
Herramientas para el desarrollo de sistemas de recuperación de información


% Pregunta 2
{\bf \question Describa el modelo espacio vectorial en Recuperación de Información (2 puntos).}

% Pregunta 3
{\bf \question Defina las medidas de evaluación Precisión y Recall en el contexto de la Recuperación de Información (2 puntos).}

% Pregunta 4
{\bf \question ¿Qué es Lucene? ¿Y Solr? (2 puntos).}

% Pregunta 5
{\bf \question Tras tres lecciones sobre Recuperación de Información, si le dijeran que tiene que diseñar un buscador de documentos ¿cuál sería la arquitectura de su sistema? No se limite al sistema de Recuperación de Información, piense que al menos necesita de una fuente de documentos, y que debe presentar los resultados de la búsqueda. (2 puntos).}

\end{questions}
	
\end{document}
