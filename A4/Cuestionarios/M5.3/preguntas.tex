\documentclass[11pt]{exam}
\usepackage[utf8]{inputenc}
\usepackage{hyperref}

\title{Evaluación del Módulo 5.3}
\author{Laura Rodríguez Navas \\ rodrigueznavas@posgrado.uimp.es}
\date{Marzo 2020}

\pagestyle{plain}

\begin{document}
	
\maketitle

\begin{questions}
	
% Pregunta 1
{\bf \question ¿Cuáles son los tipos de preguntas a los que se puede enfrentar un sistema de Búsqueda de Respuestas? (2 puntos).}

% Pregunta 2
{\bf \question ¿En qué consiste el proceso de Expansión de Consulta? (2 puntos).}

% Pregunta 3
{\bf \question ¿Qué importancia puede tener el análisis de contexto de la pregunta? Se valorará positivamente la descripción de algún ejemplo (2 puntos).}

% Pregunta 4
{\bf \question ¿Qué es un pasaje en el contexto de la Búsqueda de Respuestas? (2 puntos).}

% Pregunta 5
{\bf \question En el ámbito de la Búsqueda de Respuestas ¿la selección como respuesta de un fragmento de texto que no sea el más conciso se podría considerar como respuesta válida? Razone su respuesta. (2 puntos).}

\end{questions}
	
\end{document}
