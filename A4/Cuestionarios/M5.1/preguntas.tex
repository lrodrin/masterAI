\documentclass[11pt]{exam}
\usepackage[utf8]{inputenc}
\usepackage{hyperref}

\title{Evaluación del Módulo 5.1}
\author{Laura Rodríguez Navas \\ rodrigueznavas@posgrado.uimp.es}
\date{Marzo 2020}

\pagestyle{plain}

\begin{document}
	
\maketitle

\begin{questions}
	
% Pregunta 1
{\bf \question Defina la tarea de Extracción de Información (2 puntos).}

La Extracción de Información (EI) es un tipo de recuperación de la información cuyo objetivo es extraer automáticamente fragmentos de texto relevantes, los cuales pueden ser un nombre, un evento o una acción, en definitiva, una entidad preespecificada, y almacenar las entidades extraídas en una base de datos. En resumen, podemos decir que la extracción de información convierte información no estructurada en información estructurada. 

Típicas tareas de la extracción de información:

\begin{itemize}
	\item Reconocimiento de nombres de entidades. Esta tarea se basa en buscar, localizar y clasificar elementos atómicos (nombres) en fragmentos de texto sobre categorías predefinidas (entidades). 
	\item Resolución de la correferencia. Esta tarea está restringida a encontrar vínculos entre las entidades de los nombres que se han extraído previamente en la anterior tarea.
	\item Extracción de terminología. Esta tarea se basa identificar y extraer candidatos a términos de los fragmentos de textos explorados.
	\item Extracción de relaciones. Esta tarea se basa en detectar y clasificar menciones a relaciones semánticas.
\end{itemize}

Finalmente comentar, que la mayor aplicación de la extracción de información es la construcción de sistemas IE que encuentren y relacionen información relevante mientras ignoran otras informaciones no relevantes. Estos sistemas IE deben trabajar desde el reconocimiento de palabras hasta el análisis de frases y desde el entendimiento a nivel de frase hasta los textos completos.

% Pregunta 2
{\bf \question Atendiendo a la lección ¿es cierto que la extracción de información no requiere de la definición de la información relevante a extraer? Razone su respuesta (2 puntos).}

No es cierto. 

Como se ha comentado anteriormente, la mayor aplicación de la extracción de información es la construcción de sistemas IE que encuentren y relacionen información relevante. Para determinar si esta información es relevante, se tiene que hacer a partir de las guías predefinidas de los dominios, las cuales deben especificar con la mayor exactitud posible el tipo de información a extraer. Así, la extracción de información requiere de la definición de la información a extraer para especificar con la mayor exactitud las guías predefinidas de dominio necesarias para la construcción de los sistemas IE.

% Pregunta 3
{\bf \question Tras leer el artículo \href{https://www.aaai.org/Papers/IJCAI/2007/IJCAI07-429.pdf}{Open Information Extraction from the Web}, indique las diferencias existentes con la tarea de Extracción de Información. (4 puntos).}

Tradicionalmente, la extracción de información (IE) se ha basado en la participación humana, en forma de extracción manual de reglas o ejemplos de entrenamiento etiquetado, especificando previamente de forma explícita cada relación de interés. Porqué, aunque los sistemas IE se han vuelto cada vez más automatizados durante este tiempo, se sigue necesitando la interacción humana, en la enumeración de todas las posibles relaciones de interés para la extracción, con diferentes recorridos sobre el corpus, ya que es muy problemático para los corpus de gran tamaño y variados, como los de la Web. 

Contrariamente, en este artículo se presenta un nuevo modelo de extracción, que se adapta fácilmente a la diversidad y el tamaño de un corpus de la Web (masivo y heterogéneo). Donde el sistema IE solo hace un recorrido de extracción sobre el corpus, durante el cual las relaciones de interés se descubren automáticamente y se almacenan eficientemente sin ninguna interacción humana. 

En el pasado, los sistemas IE se han utilizado para corpus pequeños y homogéneos, como historias de noticias o anuncios de seminarios. Como resultado, los sistemas IE tradicionales pueden confiar en la lingüística "pesada" tecnológica, sintonizada con el dominio de interés, como analizadores de dependencia y reconocedores de entidades con nombre (NER). 
Pero cuando los corpus son masivos y heterogéneos, las relaciones de interés no se anticipan, y el número de relaciones suele ser grande. Esto provoca que repetidamente se incremente el coste del análisis del corpus, con el nombramiento de cada nueva relación extraída.

A diferencia de los sistemas IE tradicionales, el nuevo sistema IE presentado en el artículo, permite nombrar y explorar relaciones a velocidades más adecuadas, al mismo tiempo que se logra una mayor precisión.

% Pregunta 4
{\bf \question Atendiendo a la lección, ¿podría indicar cuáles han sido las principales conferencias relacionadas con la Extracción de Información? (2 puntos).}

La principales conferencias relacionadas con la extracción de información fueron siete. Se desarrollaron para la comprensión de mensajes y estuvieron financiadas por \href{https://es.qwe.wiki/wiki/DARPA}{DARPA}, la Agencia de Investigación Avanzada de la Defensa Proyectos, para fomentar el desarrollo de nuevos y mejores métodos de extracción de información. 

Estas conferencias tuvieron como objetivo el determinar un régimen de evaluación cuantitativo para los sistemas IE, estableciendo las tareas a llevar a cabo y el conjunto de textos sobre los que se deberían realizar dichas tareas.

Concretamente las conferencias fueron:

\begin{itemize}
	\item MUC-1 (Mayo 1987) sobre operaciones navales. \\
	Participaron 6 sistemas. \\
	No hubo definición de tareas ni se establecieron medidas de evaluación.
	\item MUC-2 (Mayo 1989) sobre operaciones navales. \\
	Participaron 8 sistemas. \\
	Se definió una plantilla y las reglas para el relleno de los diferentes atributos. \\
	Se definieron unos criterios de evaluación que se desecharon al no ser óptimos.
	\item MUC-3 (Mayo 1991) sobre atentados terroristas en América Latina. \\
	Participaron 15 sistemas. \\
	Se adaptaron los criterios de evaluación que se utilizaban en Recuperación de Información, las medidas Precison y Recall.
	\item MUC-4 (Junio 1992) sobre atentados terroristas en América Latina. \\
	Participaron 17 sistemas. \\
	Se definieron criterios de evaluación independientes de los textos, la medida F-mesure (combinación de las medidas Precison y Recall).
	\item MUC-5 (Agosto 1993) sobre fusiones de empresas y anuncios de productos microelectrónicos. \\
	Participaron 17 sistemas. \\
	Se definieron dos idiomas, el inglés y el japonés.
	\item MUC-6 (Noviembre 1995) sobre fusiones de empresas y anuncios de productos microelectrónicos. \\
	Participaron 17 sistemas. \\
	El objetivo fue la modularidad y la portabilidad de los sistemas. \\
	Se recuperaron las medidas de evaluación: Precision y Recall. \\
	Se definieron las tareas:
	\subitem - Reconocimiento de entidades (RE).
	\subitem - Resolución de correferencias (CO).
	\subitem - Plantillas de elementos (PE).
	\subitem - Plantillas de escenarios (ES).
	\item MUC-7 (Abril 1998) sobre accidentes de aviones y lanzamientos de misiles. \\
	Participaron 17 sistemas. \\
	Se definió una nueva tarea, la Relación de plantillas (RP).
\end{itemize}
	
\end{questions}
	
\end{document}
