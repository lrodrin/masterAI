\documentclass{article}
\usepackage[utf8]{inputenc}
\usepackage{natbib}
\usepackage{graphicx}
\usepackage{vmargin}
\setpapersize{A4}
\setlength{\parskip}{\baselineskip}%
\setlength{\parindent}{0pt}%

\title{Entregable Módulo de Introducción: CRISP-DM}
\author{Laura Rodríguez Navas}
\date{Marzo 2019}

\begin{document}

\maketitle

En este entregable se considera la aplicación de cada una de las fases de la metodología CRISP-DM al problema práctico que se nos plantea, que es la extracción y explotación de datos de un sistema de salud.

\section{Comprensión del Negocio}

\subsection{Determinar los Objetivos del Negocio}
El objetivo de la minería de datos que se va a aplicar en este problema práctico es el de hacer predicciones lo más fiables a partir de los atributos recogidos por un screening general de un sistema de salud que se ha realizado durante 20 años a todos los hombres cuando cumplían los 45 años. El objetivo de les predicciones es reducir el sobre-diagnóstico (falsos positivos) de cáncer de próstata, manteniendo los falsos negativos en $<$ 1\%, usando la información de muchos de estos pacientes acerca de si han sufrido o no cáncer de próstata en los años posteriores al screening.

\subsubsection{Contexto}
En referencia a la situación de negocio en el sistema de salud al principio de este problema práctico se puede decir que se cuenta con una base de datos de pacientes y existe un estudio en profundidad acerca de si han sufrido o no cáncer de próstata durante la realización del screening general y en los años posteriores al screening general del que se pueden sacar conclusiones o patrones para hacer predicciones sobre los futuros pacientes.

\subsubsection{Objetivos del negocio}
El objetivo del negocio, como ya se ha mencionado, es la reducción de los falsos positivos de pacientes de cáncer de próstata, de tal manera que se puedan hacer unas predicciones fiables partiendo de los datos que ya tenemos de dichos pacientes. Las predicciones pueden ser muy útiles para la detección i reducción de errores no deseados en las exploraciones físicas de esta enfermedad. Además permitirá al sistema de salud disminuir la cantidad de ansiedad y angustia que un falso positivo causa tanto a los médicos como a los pacientes.
	
\subsubsection{Criterios de éxito del negocio}
Desde el punto de vista del negocio se establece como criterio de éxito la posibilidad de realizar predicciones sobre los pacientes con un elevado porcentaje de fiabilidad, de tal forma que se puedan reducir los falsos positivos, manteniendo los falsos negativos en $<$ 1\%.

\subsection{Evaluación de la Situación}

\subsubsection{Inventario de recursos}\label{Inventario de recursos}
En cuanto a recursos de software disponemos del programa de minería de datos WEKA que proporciona herramientas para realizar tareas de minería de datos sobre una base de datos SQLite que es con la que contamos para el almacenamiento de los datos. Los recursos de hardware de los que disponemos son un ordenador portátil con las siguientes características:

\begin{itemize}
	\item Marca: Apple ©
	\item Modelo: MacBook Pro
	\item Procesador: Intel © Core 7 de cuatro núcleos a 3,2 GHz
	\item Memoria RAM: 8 GB
	\item Capacidad de almacenamiento: 256 GB
\end{itemize}

La fuente de datos es una base de datos SQLite, creada a partir de un fichero CSV que contiene la información de los pacientes. Cada línea del fichero CSV corresponde a un paciente diferente, que incluye su identificador y si han sufrido o no cáncer de próstata a la edad de 45 años y en los 5 años posteriores.

\subsubsection{Requisitos, supuestos y restricciones}
Al poder utilizar los datos personales de pacientes reales, esta sección no es aplicable en el proyecto.

\subsubsection{Terminología}

\begin{itemize}
	\item Falso positivo: es un error por el cual al realizar una exploración física o una prueba complementaria médica su resultado indica una enfermedad determinada, cuando en realidad no la hay.
	\item Falso negativo: es un error por el cual al realizar una exploración física o una prueba complementaria médica su resultado es normal o no detecta la alteración, cuando en realidad hay una enfermedad en el paciente.
	\item Cáncer de próstata: es un tipo de cáncer que se desarrolla en uno de los órganos glandulares del sistema reproductor masculino llamado próstata.
	\item Screening: es una estrategia utilizada para buscar afecciones o marcadores de riesgo aún no reconocidos en pacientes sin signos o síntomas.
\end{itemize}

\subsubsection{Costes y beneficios}
Los datos de este problema práctico no suponen ningún coste adicional al sistema de salud ya que estos datos pertenecen al propio sistema de salud.

En cuanto a beneficios, no se puede decir que este problema práctico genere algún beneficio económico para el sistema de salud directamente, pero sí que se puede suponer indirectamente, ya que el objetivo principal es reducir la cantidad de falsos positivos, y por tanto la satisfacción de los pacientes y los médicos aumentaría, y esto se traduce en mayor prestigio para el sistema de salud, lo cual hará que más pacientes consideren hacerse la exploración física y que más médicos trabajen en él.

\subsection{Determinar los Objetivos de la Minería de Datos}\label{Objetivos de la Minería de Datos}
Los objetivos en términos de minería de datos son:

\begin{itemize}
	\item Predecir falsos positivos para reducir el sobre-diagnóstico de cáncer de próstata de futuros pacientes.
	\item Mantener las predicciones de falsos negativos en futuros pacientes inferiores al 1\%, acorde a la predicción de falsos positivos.
\end{itemize}

\subsubsection{Criterios de éxito de minería de datos}
Desde el punto de vista de la minería de datos se establece como criterio de éxito la posibilidad de realizar predicciones sobre los pacientes con un elevado porcentaje de fiabilidad, concretamente se podría definir este porcentaje en un 80\%. Aunque el grado de fiabilidad lo determinará el algoritmo específico que se emplee a la hora de conseguir el modelo de la minería de datos, por lo que este tema se aborda más adelante en la metodología (evaluación).

\subsection{Realizar el Plan del Proyecto}
El proyecto se dividirá en las siguientes etapas para facilitar su organización y estimar el tiempo de realización del mismo:

\begin{itemize}
	\item Etapa 1: Análisis de la estructura de los datos y la información de la base de datos. Tiempo estimado: 2 semanas.
	\item Etapa 2: Ejecución de consultas para tener muestras representativas de los datos. Tiempo estimado: 1 semana.
	\item Etapa 3: Preparación de los datos (selección, limpieza, conversión y formateo, si fuera necesario) para facilitar la minería de datos sobre ellos. Tiempo estimado: 3 semanas.
	\item Etapa 4: Elección de las técnicas de modelado y ejecución de las mismas sobre los datos. Tiempo estimado: 1 semana.
	\item Etapa 5: Análisis de los resultados obtenidos en la etapa anterior, si fuera necesario repetir la etapa 4. Tiempo estimado: 1 semana.
	\item Etapa 6: Producción de informes con los resultados obtenidos en función de los objetivos de negocio y los criterios de éxito establecidos. Tiempo estimado: 1 semana.
	\item Etapa 7: Presentación de los resultados finales. Tiempo estimado: 1 semana.
\end{itemize}

\subsubsection{Evaluación inicial de herramientas y técnicas}\label{Evaluación inicial de herramientas y técnicas}
La herramienta que se va a utilizar para llevar a cabo este proyecto de minería de datos es WEKA, como ya se comentó en el apartado \ref{Inventario de recursos}. 

Se podría añadir que WEKA, aparte de ser un programa de código abierto, ofrece una colección de herramientas de visualización y algoritmos para el análisis de datos y modelado predictivo, considerado para llevar a cabo este proyecto de minería de datos.  

\section{Comprensión de los Datos}

\subsection{Recolectar los Datos Iniciales}
Los datos utilizados en este proyecto son datos referentes a pacientes que incluyen información personal sobre ellos; y como se utilizan datos reales de pacientes existentes en el sistema de salud, las predicciones y estudios en el proyecto serán muy realistas.

Aunque, debido a la gran cantidad de pacientes que son necesarios para poder hacer el proyecto de minería de datos con éxito, la opción de insertar los pacientes manualmente uno a uno en la base de datos no sería viable, por lo que se optaría por crear un programa de inserción de datos en el lenguaje de programación Python, cuya salida fuera una tabla para la base de datos construida a partir del fichero CSV de entrada adquirido.

\subsection{Descripción de los Datos}
Los datos se encontrarán almacenados en una tabla SQLite, llamada Pacientes. Cada fila de la tabla corresponde a un único paciente, y cada paciente estará identificado por su número de paciente, que es un valor numérico y único. También se indicará con un valor booleano, si los pacientes han sufrido (T) o no (F) cáncer de próstata a la edad de 45 años y en los 5 años posteriores.

Por ejemplo, la tabla Pacientes se estructuraría:

\begin{center}
	\begin{tabular}{ |c|c|c| } 
		\hline
		IDPaciente & 45 & +45 \\
		\hline
		... & ... & ... \\ 
		5002 & T & T\\ 
		5003 & F & T\\ 
		5004 & F & F\\
		... & ... & ... \\ 
		\hline
\end{tabular}
\end{center}

\subsection{Exploración de los Datos}

\subsection{Verificar la Calidad de los Datos}
Después de hacer la exploración inicial de los datos se puede afirmar que estos serían completos. Los datos cubren los casos requeridos para la obtención de los resultados necesarios para poder cumplir los objetivos del proyecto y no contienen errores, ya que son datos generados automáticamente por el script de inserción de datos. Tampoco se encontrarían valores fuera de rango, ya que los datos son controlados desde el mismo script, por lo que no hay riesgo de ruido en el proceso de la minería de datos.

\section{Preparación de los Datos}

\subsection{Seleccionar los Datos}
En términos de registros, para el análisis se van a utilizar todos los campos de IDPaciente, 45 y +45, dentro de la tabla Pacientes que compone la base de datos, ya que al ser ésta una base de datos específicamente creada para este proyecto, el número de pacientes que se han insertado ha sido elegido a propósito. 

\subsection{Limpiar los Datos}
La base de datos con la que se cuenta para el proyecto contiene toda la información necesaria para poder cumplir los objetivos de la minería de datos, además, estos datos al haber sido introducidos ex profeso para el caso práctico que se presenta, son datos limpios que no contienen valores nulos y por lo tanto no hay necesidad de hacer una limpieza sobre ellos.

\subsection{Construir los Datos}\label{Construir los Datos}
No seria aplicable la construcción de datos en este proyecto porqué no es necesario que se realicen operaciones de transformación sobre atributos derivados de los datos. Tampoco será necesario generar nuevos atributos ni nuevos registros sobre la base de datos, ya que está completa y ha sido creada específicamente para su uso en este proyecto.

\subsection{Integrar los Datos}
Como se acaba de comentar en el apartado \ref{Construir los Datos}, no seria necesaria la creación de nuevas estructuras (atributos, registros, etc.), ni la fusión entre distintas tablas de la base de datos ya que solo existe una tabla.

\subsection{Formateo de los Datos}
No será necesario cambiar el orden de ningún campo dentro de los registros, ni tampoco la re-ordenación de los registros dentro de la tabla. Tampoco será necesario cambiar el formato de ninguno de los campos que se van a utilizar para la minería de datos ya que el formato actual es admitido por la herramienta WEKA.

\section{Modelado}

\subsection{Escoger la Técnica de Modelado}
Debido a que se va a utilizar el software WEKA para realizar la minería de datos, se utilizará alguna de las técnicas de modelado que nos ofrece esta herramienta de acuerdo con los objetivos del proyecto que están reflejados en el apartado \ref{Objetivos de la Minería de Datos}.

De los modelos que nos ofrece WEKA, el que mejor se adapta a nuestros objetivos sería un modelo de regresión, puesto que los problemas que queremos resolver son problemas de predicción y los campos que se quieren predecir contienen valores continuos.

\subsection{Generar el Plan de Prueba}
\subsection{Construir el Modelo}
\subsection{Evaluar el Modelo}

\section{Evaluación}
\subsection{Evaluar los Resultados}
\subsection{Revisar el Proceso}
\subsection{Determinar los Próximos Pasos}
El siguiente paso a realizar en el proyecto es el de ejecutar la etapa de implantación para los objetivos 1 y 2.

\section{Implantación}
\subsection{Planear la Implantación}
\subsection{Planear la Monitorización y el Mantenimiento}
\subsection{Producir el Informe Final}
\subsection{Revisar el Proyecto}
\end{document}
