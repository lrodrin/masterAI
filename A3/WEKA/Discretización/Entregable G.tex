\documentclass{article}
\usepackage[utf8]{inputenc}
\usepackage{natbib}
\usepackage{graphicx}
\usepackage{vmargin}
\usepackage{hyperref}
\usepackage{booktabs}
\usepackage{multirow}
\usepackage{siunitx}
\setpapersize{A4}
\setlength{\parskip}{\baselineskip}%
\setlength{\parindent}{0pt}%

\title{Entregable Discretización}
\author{Laura Rodríguez Navas \\ rodrigueznavas@posgrado.uimp.es}
\date{Marzo 2020}

\begin{document}

\maketitle

El objetivo de esta práctica es analizar sobre la base de datos considerada, el efecto de incluir o no métodos de preprocesamiento, con los algoritmos de aprendizaje kNN y perceptrón multicapa, para mejorar las medidas iniciales de estos. Para ello se usa una validación cruzada de 5 carpetas (5cv) y la herramienta \href{https://www.cs.waikato.ac.nz/ml/weka/}{WEKA}.

\section*{Exploración de los datos}

Consideramos la base de datos \href{https://github.com/renatopp/arff-datasets/blob/master/classification/vehicle.arff}{vehicle} definida sobre 18 variables predictivas (todas numéricas) y una variable clase multiclase \{opel, saab, bus, van\}. En ella no existen valores desconocidos, no está ordenada en función de la variable clase y está formada por 846 registros. 

La distribución de la variable clase no es totalmente uniforme. Cuenta con:

\begin{itemize}
	\item 212 registros para la etiqueta de la variable clase opel.
	\item 217 registros para la etiqueta de la variable clase saab.
	\item 218 registros para la etiqueta de la variable clase bus.
	\item 199 registros para la etiqueta de la variable clase van.
\end{itemize}

A continuación, se muestra la gráfica de la distribución de la variable clase.

GRÁFICA

Después usamos un filtro de tipo no supervisado y de registro, llamado RemoveFolds proporcionado por la herramienta WEKA, que elimina alguna de las carpetas en una validación cruzada, para crear un conjunto de entrenamiento que contendrá dos tercios de los registros de la base de datos. Una vez aplicado, el número de registros se reduce a 564.

La distribución de la variable clase sigue siendo no uniforme y ahora cuenta con:

\begin{itemize}
	\item 149 registros para la etiqueta de la variable clase opel.
	\item 153 registros para la etiqueta de la variable clase saab.
	\item 139 registros para la etiqueta de la variable clase bus.
	\item 123 registros para la etiqueta de la variable clase van.
\end{itemize}

A continuación, se muestra la gráfica de la distribución de la variable clase del conjunto de entrenamiento.

GRÁFICA

\section*{Algoritmos de preprocesamiento}

Los algoritmos que se describen a continuación son:

\begin{itemize}
	\item
	\item
	\item
\end{itemize}

\section*{Resultados y análisis}

Se han considerado dos parámetros de rendimiento para la evaluación de los resultados. Los siguientes parámetros son examinados tanto antes como después de la discretización: Accuracy y Error Rate. 

La siguiente tabla proporciona la precisión y la tasa de error para cada clasificador antes y después de la discretización.

\begin{center}
	\begin{tabular}{cSSSSS}
		\toprule
		\multirow{2}{*}{Clasificador} &
		\multicolumn{2}{c}{Antes Disc.} &
		\multicolumn{2}{c}{Después Disc.} \\
		& {Acc in \%} & {ERR in \%} & {Acc in \%} & {ERR in \%} \\
		\midrule
		kNN (k=1) & 64.7163 & 35.2837 & 2.1 & 2.1 \\
		kNN (k=3) & 66.844 & 33.156 & 2.1 & 2.1 \\
		Multilayer Perceptron & 79.4326 & 20.5674 & 11.6 & 11.6 \\
		\bottomrule
	\end{tabular}
\end{center}

\end{document}
