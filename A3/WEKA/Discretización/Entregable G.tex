\documentclass{article}
\usepackage[utf8]{inputenc}
\usepackage{natbib}
\usepackage{graphicx}
\usepackage{vmargin}
\usepackage{hyperref}
\setpapersize{A4}
\setlength{\parskip}{\baselineskip}%
\setlength{\parindent}{0pt}%

\title{Entregable Discretización}
\author{Laura Rodríguez Navas \\ rodrigueznavas@posgrado.uimp.es}
\date{Marzo 2020}

\begin{document}

\maketitle

\section*{Preprocesamiento}

Después de analizar la base de datos, observamos que en ella no existen valores desconocidos. Como consecuencia, no tendremos que aplicar ninguna técnica de preprocesamiento para los valores desconocidos en esta base de datos. También observamos que la base de datos esta ordenada en función de las variables de la clase \{Tumor, Normal\}. En este caso, sí que tendremos que aplicar una técnica de preprocesamiento y procederemos a aleatorizar la base de datos. Para desordenar la base de datos aplicaremos un filtro a nivel de registro, concretamente de tipo no supervisado y de tipo registro llamado Ramdomize. Utilizaremos la semilla que viene por defecto. 

A continuación, dividimos la base de datos en un conjunto de entrenamiento con dos tercios de los registros, y uno de test, con un tercio de los registros. Además lo haremos sin remplazo. Así, lo que tenemos que utilizar para ello, es un filtro de tipo no supervisado y de tipo registro llamado RemoveFolds. Los valores de este filtro que debemos modificar son: numFols = 3, fold = 1 y invertSelection = True. Como resultado hemos creado un conjunto de entrenamiento con 90 registros.

\section*{Clasificación}

\section*{Selección de atributos}

\subsection*{NaiveBayes}

\begin{center}
	\begin{tabular}{ |c|c|c|c| } 
		\hline
		col & Accuracy & F-measure \\
		\hline
		NaiveBayes& cell5 & cell6 \\ 
		C4.5 & cell8 & cell9 \\ 
		\hline
	\end{tabular}
\end{center}

\subsection*{J48 (C4.5)}

\section*{Técnicas de preprocesamiento}

Selección de variables

AttributeSelectedClassifier

FilteredClassifier

\section*{Visualitzation}

Confusion matrix
ROC




1. inicialmente lanzaremos los clasificadores sobre la base de datos inicial, obteniendo así una medida inicial que intentaremos mejorar.

2. mejorar con la selección de variables

de variables es no lanzar una selección de tipo multivariada (filter o wrapper) sobre todas las variables, puesto que podría llevar mucho tiempo. Se recomienda primero hacer una selección univariada tipo ranker y quedarse con los mejores 200-500 atributos.


\end{document}
