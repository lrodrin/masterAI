\documentclass{article}
\usepackage[utf8]{inputenc}
\usepackage{natbib}
\usepackage{graphicx}
\usepackage{hyperref}

\title{Entregable WEKA}
\author{Laura Rodríguez Navas}
\date{Marzo 2020}

\begin{document}

\maketitle

\section*{Preparación de Datos}

Quitar valores missing

aleatorizar la base de datos

crear conjunto de entrenamiento

\section*{Uso de los clasificadores}

cross-validation (5cv)

\subsection*{NaiveBayes}

\begin{center}
	\begin{tabular}{ |c|c|c|c| } 
		\hline
		col & Accuracy & F-measure \\
		\hline
		NaiveBayes& cell5 & cell6 \\ 
		C4.5 & cell8 & cell9 \\ 
		\hline
	\end{tabular}
\end{center}

\subsection*{J48 (C4.5)}

\section*{Técnicas de preprocesamiento}

Selección de variables

AttributeSelectedClassifier

FilteredClassifier

\section*{Visualitzation}

Confusion matrix
ROC




1. inicialmente lanzaremos los clasificadores sobre la base de datos inicial, obteniendo así una medida inicial que intentaremos mejorar.

2. mejorar con la selección de variables

de variables es no lanzar una selección de tipo multivariada (filter o wrapper) sobre todas las variables, puesto que podría llevar mucho tiempo. Se recomienda primero hacer una selección univariada tipo ranker y quedarse con los mejores 200-500 atributos.


\end{document}
