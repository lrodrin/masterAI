\documentclass{article}
\usepackage[utf8]{inputenc}
\usepackage{natbib}
\usepackage{graphicx}
\usepackage{vmargin}
\usepackage{hyperref}
\usepackage{booktabs}
\usepackage{multirow}
\usepackage{siunitx}
\setpapersize{A4}
\setlength{\parskip}{\baselineskip}%
\setlength{\parindent}{0pt}%

\title{Entregable WEKA}
\author{Laura Rodríguez Navas \\ rodrigueznavas@posgrado.uimp.es}
\date{Mayo 2020}

\begin{document}

\maketitle

\section*{Preparación de datos}

Consideramos la base de datos Prostate definida sobre 12600 variables predictivas (todas numéricas) y una variable clase binaria \{Tumor, Normal\}. Está formada por 136 registros y en ella no existen valores desconocidos. Pero está ordenada en función de la variable clase. Como consecuencia, tenemos que aleatorizar la base de datos. Para ello se aplica un filtro a nivel de registro, concretamente de tipo no supervisado llamado Ramdomize. Usamos la semilla que viene por defecto (42).

Después, dividimos la base de datos en un conjunto de entrenamiento, con dos tercios de los registros, y en un conjunto de prueba, con un tercio de los registros. Para ello se aplica un filtro a nivel de registro y no supervisado llamado RemoveFolds. Como resultado hemos creado un conjunto de entrenamiento con 90 registros.

\section*{Clasificación}

Se usan los clasificadores Naive Bayes y J48 (C4.5) en una validación cruzada de 5 carpetas (5cv) sobre el conjunto de entrenamiento que acabamos de crear.

Se han considerado dos parámetros de rendimiento para la evaluación de los resultados de las clasificaciones. Los parámetros examinados antes y después de una discretización y/o de una selección de variables son registros clasificados correctamente (Accuracy) y los registros no clasificados correctamente (Error Rate). 

Después de realizar una primera clasificación obtenemos los siguientes resultados:

\begin{center}
	\begin{tabular}{ |c|c|c| } 
		\hline
		Clasificador & Acc. en \% & ERR en \% \\
		\hline
		Naive Bayes & 52.2222 & 47.7778 \\ 
		J48 (C4.5) & 82.2222 & 17.7778 \\ 
		\hline
	\end{tabular}
\end{center}

Inicialmente y sin la aplicación de algún tipo mejora, podemos observar que el algoritmo J48 (C4.5) clasifica mucho mejor, en comparación con el algoritmo Naive Bayes. Se puede pensar que se debe a la alta presencia de variables relevantes y redundantes en la base de datos, ya que sabemos que Naive Bayes es un algoritmo sensible a este caso.

\section*{Mejoras}

Las mejoras se basan en técnicas de discretización, de selección de variables y una combinación de ambas.

\subsection*{Discretización}

Primero probamos un método supervisado a nivel de atributo basado en la entropía, concretamente usamos MLDP. Como resultado, los valores de Accuracy y Error Rate después de esta discretización son:

\begin{center}
	\begin{tabular}{cSSSSS}
		\toprule
		\multirow{2}{*}{Clasificador} &
		\multicolumn{2}{c}{Antes Disc.} &
		\multicolumn{2}{c}{Después Disc.} \\
		& {Acc. in \%} & {ERR in \%} & {Acc. in \%} & {ERR in \%} \\
		\midrule
		Naives Bayes & 52.2222 & 47.7778 & 82.2222 & 17.7778 \\
		J48 (C4.5) & 82.2222 & 17.7778 & 87.7778 & 12.2222 \\
		\bottomrule
	\end{tabular}
\end{center}

Observamos una mejora después de la discretización en los dos algoritmos, y a continuación, probamos dos métodos no supervisados a nivel de atributo.

\begin{center}
	\begin{itemize}
		\item Intervalos de igual amplitud. Es el método de discretización no supervisado más simple, que determina los valores mínimo y máximo del atributo discretizado y luego divide el rango en el número definido por el usuario de intervalos discretos de igual amplitud. La tabla siguiente contiene los valores de precisión y tasa de error que dependen del número de bins utilizados (2, 4, 5 y 10). 
		\begin{center}
			\begin{tabular}{cSSSSS}
				\toprule
				\multirow{2}{*}{\# of bins} &
				\multicolumn{2}{c}{Naives Bayes} &
				\multicolumn{2}{c}{J48 (C4.5)} \\
				& {Acc. in \%} & {ERR in \%} & {Acc. in \%} & {ERR in \%} \\
				\midrule
				2 & 51.1111 & 48.8889 & 63.333 & 36.6667 \\
				4 & 54.4444 & 45.5556 & 72.2222 & 27.7778 \\
				5 & 57.7778 & 42.2222 & 68.8889 & 31.1111 \\
				10 & 71.1111 & 28.8889 & 73.3333 & 26.6667 \\
				\bottomrule
			\end{tabular}				
		\end{center}
		Observamos que el algoritmo J48 (C4.5) no mejora después de la discretización y el algoritmo Naive Bayes empieza a mejorar cuando el número de bins es superior a 2.
		\item Intervalos de igual frecuencia. Es el método no supervisado que divide los valores ordenados en k intervalos para que cada intervalo contenga aproximadamente el mismo número de registros de entrenamiento. Por lo tanto, cada intervalo contiene n/k (posiblemente duplicados) valores adyacentes. Aquí k representa el número de bins. La tabla siguiente contiene los valores de precisión y tasa de error que dependen del número de bins utilizados (2, 4, 5 y 10). 
		\begin{center}
			\begin{tabular}{cSSSSS}
				\toprule
				\multirow{2}{*}{\# of bins} &
				\multicolumn{2}{c}{Naives Bayes} &
				\multicolumn{2}{c}{J48 (C4.5)} \\
				& {Acc. in \%} & {ERR in \%} & {Acc. in \%} & {ERR in \%} \\
				\midrule
				2 & 65.5556 & 34.4444 & 71.1111 & 28.8889 \\
				4 & 65.5556 & 34.4444 & 80 & 20 \\
				5 & 70 & 30 & 84.4444 & 15.5556 \\
				10 & 72.2222 & 27.7778 & 70 & 30 \\
				\bottomrule
			\end{tabular}
		\end{center}
		Observamos que el algoritmo J48 (C4.5) tampoco mejora después de la discretización y el algoritmo Naive Bayes, en cambio, mejora bastante.	
	\end{itemize}
\end{center}

Como conclusión, comentar que escogemos el método supervisado como mejor discretización, porqué es un método que realiza mejor la selección de intervalos al ser un método supervisado, y además sin la necesidad de hacer pruebas con distintos intervalos.

Así que utilizaremos la siguiente tabla para comparar, las técnicas de discretización utilizadas con las técnicas de selección de variables que analizaremos a continuación.

\begin{center}
	\begin{tabular}{cSSSSS}
		\toprule
		\multirow{2}{*}{Clasificador} &
		\multicolumn{2}{c}{Antes Disc.} &
		\multicolumn{2}{c}{Después Disc.} \\
		& {Acc. in \%} & {ERR in \%} & {Acc. in \%} & {ERR in \%} \\
		\midrule
		Naives Bayes & 52.2222 & 47.7778 & 82.2222 & 17.7778 \\
		J48 (C4.5) & 82.2222 & 17.7778 & 87.7778 & 12.2222 \\
		\bottomrule
	\end{tabular}
\end{center}

\subsection*{Selección de variables}

Sabemos que la base de datos contiene muchos atributos, así que primero reduciremos el número de atributos para quedarnos con los 200 más representativos y la variable clase, haciendo una selección invariada tipo ranker. Después discretizamos con el método supervisado utilizado anteriormente. Hecho esto, es significativo como el algoritmo Naive Bayes mejora muchísimo después de la discretización. Contrariamente, el clasificador J48 (C4.5) empeora. Eso demuestra que al reducir el número de atributos, reducimos el número de variables relevantes y redundantes en la base de datos; y que Naive Bayes es un algoritmo sensible a este caso y que el algoritmo J48 (C4.5) no lo es.

La siguiente tabla corrobora el comentario anterior.

\begin{center}
	\begin{tabular}{cSSSSS}
		\toprule
		\multirow{2}{*}{Clasificador} &
		\multicolumn{2}{c}{Sin Disc.} &
		\multicolumn{2}{c}{Con Disc.} \\
		& {Acc. in \%} & {ERR in \%} & {Acc. in \%} & {ERR in \%} \\
		\midrule
		Naives Bayes & 52.2222 & 47.7778 & 80 & 20 \\
		J48 (C4.5) & 91.1111 & 8.8889 & 85.5556 & 14.4444 \\
		\bottomrule
	\end{tabular}
\end{center}

Otra técnica de mejora que podemos utilizar es una selección multivariada de tipo Filter, caracterizada por una técnica de búsqueda CFS (Correlation Feature Subset), y una función de evaluación en selección hacia delante (forward). En este caso la selección de atributos se ve ampliamente reducida a 18 atributos. También se prueba con discretización y sin discretización.

\begin{center}
	\begin{tabular}{cSSSSS}
		\toprule
		\multirow{2}{*}{Clasificador} &
		\multicolumn{2}{c}{Sin Disc.} &
		\multicolumn{2}{c}{Con Disc.} \\
		& {Acc. in \%} & {ERR in \%} & {Acc. in \%} & {ERR in \%} \\
		\midrule
		Naives Bayes & 58.8889 & 41.1111 & 96.6667 & 3.3333 \\
		J48 (C4.5) & 88.8889 & 11.1111 & 83.3333 & 16.6667 \\
		\bottomrule
	\end{tabular}
\end{center}

No mejora en ningún caso para el algoritmo J48 (C4.5), sí que lo hace para el algoritmo Naive Bayes.

Finalmente, probamos una selección multivariada de tipo Wrapper para cada clasificador, con la función de evaluación en selección hacia delante (forward). También se prueba con discretización y sin discretización.

\begin{center}
	\begin{itemize}
		\item Naive Bayes. La selección de atributos se reduce a 2 \{67, 174\}. \\
		\begin{center}
			\begin{tabular}{cSSSSS}
				\toprule
				\multirow{2}{*}{Clasificador} &
				\multicolumn{2}{c}{Sin Disc.} &
				\multicolumn{2}{c}{Con Disc.} \\
				& {Acc. in \%} & {ERR in \%} & {Acc. in \%} & {ERR in \%} \\
				\midrule
				Naives Bayes & 83.3333 & 16.6667 & 82.2222 & 17.7778 \\
				\bottomrule
			\end{tabular}
		\end{center}
		\item J48 (C4.5). La selección de atributos se reduce a 3 \{1, 52, 166\}. \\
		\begin{center}
			\begin{tabular}{cSSSSS}
				\toprule
				\multirow{2}{*}{Clasificador} &
				\multicolumn{2}{c}{Sin Disc.} &
				\multicolumn{2}{c}{Con Disc.} \\
				& {Acc. in \%} & {ERR in \%} & {Acc. in \%} & {ERR in \%} \\
				\midrule
				J48 (C4.5) & 92.2222 & 7.7778 & 86.6667 & 13.3333 \\
				\bottomrule
			\end{tabular}
		\end{center}
	\end{itemize} 
\end{center}

De las tres técnicas de selección de variables, nos quedamos con la última que és la mejor (con y sin discretización).

\end{document}
