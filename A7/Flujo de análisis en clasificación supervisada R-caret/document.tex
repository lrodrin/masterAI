% Options for packages loaded elsewhere
\PassOptionsToPackage{unicode}{hyperref}
\PassOptionsToPackage{hyphens}{url}
%
\documentclass[
]{article}
\usepackage{lmodern}
\usepackage{amssymb,amsmath}
\usepackage{ifxetex,ifluatex}
\ifnum 0\ifxetex 1\fi\ifluatex 1\fi=0 % if pdftex
  \usepackage[T1]{fontenc}
  \usepackage[utf8]{inputenc}
  \usepackage{textcomp} % provide euro and other symbols
\else % if luatex or xetex
  \usepackage{unicode-math}
  \defaultfontfeatures{Scale=MatchLowercase}
  \defaultfontfeatures[\rmfamily]{Ligatures=TeX,Scale=1}
\fi
% Use upquote if available, for straight quotes in verbatim environments
\IfFileExists{upquote.sty}{\usepackage{upquote}}{}
\IfFileExists{microtype.sty}{% use microtype if available
  \usepackage[]{microtype}
  \UseMicrotypeSet[protrusion]{basicmath} % disable protrusion for tt fonts
}{}
\makeatletter
\@ifundefined{KOMAClassName}{% if non-KOMA class
  \IfFileExists{parskip.sty}{%
    \usepackage{parskip}
  }{% else
    \setlength{\parindent}{0pt}
    \setlength{\parskip}{6pt plus 2pt minus 1pt}}
}{% if KOMA class
  \KOMAoptions{parskip=half}}
\makeatother
\usepackage{xcolor}
\IfFileExists{xurl.sty}{\usepackage{xurl}}{} % add URL line breaks if available
\IfFileExists{bookmark.sty}{\usepackage{bookmark}}{\usepackage{hyperref}}
\hypersetup{
  pdftitle={Flujo de análisis en clasificación supervisada},
  pdfauthor={Laura Rodríguez Navas},
  hidelinks,
  pdfcreator={LaTeX via pandoc}}
\urlstyle{same} % disable monospaced font for URLs
\usepackage[margin=1in]{geometry}
\usepackage{color}
\usepackage{fancyvrb}
\newcommand{\VerbBar}{|}
\newcommand{\VERB}{\Verb[commandchars=\\\{\}]}
\DefineVerbatimEnvironment{Highlighting}{Verbatim}{commandchars=\\\{\}}
% Add ',fontsize=\small' for more characters per line
\usepackage{framed}
\definecolor{shadecolor}{RGB}{248,248,248}
\newenvironment{Shaded}{\begin{snugshade}}{\end{snugshade}}
\newcommand{\AlertTok}[1]{\textcolor[rgb]{0.94,0.16,0.16}{#1}}
\newcommand{\AnnotationTok}[1]{\textcolor[rgb]{0.56,0.35,0.01}{\textbf{\textit{#1}}}}
\newcommand{\AttributeTok}[1]{\textcolor[rgb]{0.77,0.63,0.00}{#1}}
\newcommand{\BaseNTok}[1]{\textcolor[rgb]{0.00,0.00,0.81}{#1}}
\newcommand{\BuiltInTok}[1]{#1}
\newcommand{\CharTok}[1]{\textcolor[rgb]{0.31,0.60,0.02}{#1}}
\newcommand{\CommentTok}[1]{\textcolor[rgb]{0.56,0.35,0.01}{\textit{#1}}}
\newcommand{\CommentVarTok}[1]{\textcolor[rgb]{0.56,0.35,0.01}{\textbf{\textit{#1}}}}
\newcommand{\ConstantTok}[1]{\textcolor[rgb]{0.00,0.00,0.00}{#1}}
\newcommand{\ControlFlowTok}[1]{\textcolor[rgb]{0.13,0.29,0.53}{\textbf{#1}}}
\newcommand{\DataTypeTok}[1]{\textcolor[rgb]{0.13,0.29,0.53}{#1}}
\newcommand{\DecValTok}[1]{\textcolor[rgb]{0.00,0.00,0.81}{#1}}
\newcommand{\DocumentationTok}[1]{\textcolor[rgb]{0.56,0.35,0.01}{\textbf{\textit{#1}}}}
\newcommand{\ErrorTok}[1]{\textcolor[rgb]{0.64,0.00,0.00}{\textbf{#1}}}
\newcommand{\ExtensionTok}[1]{#1}
\newcommand{\FloatTok}[1]{\textcolor[rgb]{0.00,0.00,0.81}{#1}}
\newcommand{\FunctionTok}[1]{\textcolor[rgb]{0.00,0.00,0.00}{#1}}
\newcommand{\ImportTok}[1]{#1}
\newcommand{\InformationTok}[1]{\textcolor[rgb]{0.56,0.35,0.01}{\textbf{\textit{#1}}}}
\newcommand{\KeywordTok}[1]{\textcolor[rgb]{0.13,0.29,0.53}{\textbf{#1}}}
\newcommand{\NormalTok}[1]{#1}
\newcommand{\OperatorTok}[1]{\textcolor[rgb]{0.81,0.36,0.00}{\textbf{#1}}}
\newcommand{\OtherTok}[1]{\textcolor[rgb]{0.56,0.35,0.01}{#1}}
\newcommand{\PreprocessorTok}[1]{\textcolor[rgb]{0.56,0.35,0.01}{\textit{#1}}}
\newcommand{\RegionMarkerTok}[1]{#1}
\newcommand{\SpecialCharTok}[1]{\textcolor[rgb]{0.00,0.00,0.00}{#1}}
\newcommand{\SpecialStringTok}[1]{\textcolor[rgb]{0.31,0.60,0.02}{#1}}
\newcommand{\StringTok}[1]{\textcolor[rgb]{0.31,0.60,0.02}{#1}}
\newcommand{\VariableTok}[1]{\textcolor[rgb]{0.00,0.00,0.00}{#1}}
\newcommand{\VerbatimStringTok}[1]{\textcolor[rgb]{0.31,0.60,0.02}{#1}}
\newcommand{\WarningTok}[1]{\textcolor[rgb]{0.56,0.35,0.01}{\textbf{\textit{#1}}}}
\usepackage{graphicx,grffile}
\makeatletter
\def\maxwidth{\ifdim\Gin@nat@width>\linewidth\linewidth\else\Gin@nat@width\fi}
\def\maxheight{\ifdim\Gin@nat@height>\textheight\textheight\else\Gin@nat@height\fi}
\makeatother
% Scale images if necessary, so that they will not overflow the page
% margins by default, and it is still possible to overwrite the defaults
% using explicit options in \includegraphics[width, height, ...]{}
\setkeys{Gin}{width=\maxwidth,height=\maxheight,keepaspectratio}
% Set default figure placement to htbp
\makeatletter
\def\fps@figure{htbp}
\makeatother
\setlength{\emergencystretch}{3em} % prevent overfull lines
\providecommand{\tightlist}{%
  \setlength{\itemsep}{0pt}\setlength{\parskip}{0pt}}
\setcounter{secnumdepth}{-\maxdimen} % remove section numbering

\title{Flujo de análisis en clasificación supervisada}
\usepackage{etoolbox}
\makeatletter
\providecommand{\subtitle}[1]{% add subtitle to \maketitle
  \apptocmd{\@title}{\par {\large #1 \par}}{}{}
}
\makeatother
\subtitle{Métodos supervisados}
\author{Laura Rodríguez Navas}
\date{Septiembre 2020}

\begin{document}
\maketitle

{
\setcounter{tocdepth}{2}
\tableofcontents
}
Comenzamos cargando los paquetes necesarios.

\begin{Shaded}
\begin{Highlighting}[]
\KeywordTok{library}\NormalTok{(tidyverse)}
\KeywordTok{library}\NormalTok{(stringi)}
\KeywordTok{library}\NormalTok{(tm)}
\KeywordTok{library}\NormalTok{(irlba)}
\KeywordTok{library}\NormalTok{(RColorBrewer)}
\KeywordTok{library}\NormalTok{(wordcloud)}
\KeywordTok{library}\NormalTok{(gridExtra)}
\KeywordTok{library}\NormalTok{(caret)}
\KeywordTok{library}\NormalTok{(doParallel)}
\KeywordTok{library}\NormalTok{(syuzhet)}
\KeywordTok{library}\NormalTok{(ggcorrplot)}
\end{Highlighting}
\end{Shaded}

\hypertarget{anuxe1lisis-exploratorio-de-los-datos}{%
\subsection{Análisis Exploratorio de los
Datos}\label{anuxe1lisis-exploratorio-de-los-datos}}

Para la realización del ejercicio propuesto se ha elegido la competición
en Kaggle: \textbf{Real or Not? NLP with Disaster Tweets}. El dataset de
la competición se puede encontrar en el siguiente enlace:
\url{https://www.kaggle.com/c/nlp-getting-started/data}. Este dataset,
con 10.876 instancias, contiene 4 variables explicativas: \textbf{id},
\textbf{keyword}, \textbf{location} y \textbf{text}, y dos valores en la
variable clase \textbf{target} (0 y 1). La variable clase es binaria,
así que, vamos a aprender un modelo de clasificación binaria. El
objetivo de este modelo será predecir si dado un tweet, este tweet trata
sobre un desastre real o no. Si un tweet trata sobre un desastre real,
se predice un 1. Si no, se predice un 0.

La métrica de evaluación esperada por la competición es
\href{https://www.kaggle.com/c/nlp-getting-started/overview/evaluation}{F1}.
Y se calcula de la siguiente manera:

\begin{center}\includegraphics[width=0.3\linewidth]{F1_score} \end{center}

donde:

\begin{center}\includegraphics[width=0.25\linewidth]{F1_score_2} \end{center}

La partición inicial train-test, no se tiene que realizar, ya que las
instancias de train y test ya vienen definidas en el dataset de la
competición (ficheros \textbf{train.csv} y \textbf{test.csv}).

A continuación, cargaremos el conjunto de datos de train y test,
nombrando los valores perdidos como \textbf{NA} para que los podamos
tratar más adelante, y mostraremos sus dimensiones.

\begin{Shaded}
\begin{Highlighting}[]
\NormalTok{train <-}\StringTok{ }\KeywordTok{read.csv}\NormalTok{(}\StringTok{"train.csv"}\NormalTok{, }\DataTypeTok{na.strings=}\KeywordTok{c}\NormalTok{(}\StringTok{""}\NormalTok{, }\StringTok{"NA"}\NormalTok{))}
\NormalTok{test <-}\StringTok{ }\KeywordTok{read.csv}\NormalTok{(}\StringTok{"test.csv"}\NormalTok{, }\DataTypeTok{na.strings=}\KeywordTok{c}\NormalTok{(}\StringTok{""}\NormalTok{, }\StringTok{"NA"}\NormalTok{))}
\KeywordTok{dim}\NormalTok{(train)}
\end{Highlighting}
\end{Shaded}

\begin{verbatim}
## [1] 7613    5
\end{verbatim}

\begin{Shaded}
\begin{Highlighting}[]
\KeywordTok{dim}\NormalTok{(test)}
\end{Highlighting}
\end{Shaded}

\begin{verbatim}
## [1] 3263    4
\end{verbatim}

El conjunto de datos de train contiene 7613 instancias y el conjunto de
datos de test contiene 3263 instancias. Cada instancia de estos
conjuntos contiene la siguiente información:

\begin{itemize}
\tightlist
\item
  \textbf{id}: un identificador único para cada tweet.
\item
  \textbf{keyword}: una palabra clave del tweet.
\item
  \textbf{location}: la ubicación desde la que se envió el tweet.
\item
  \textbf{text}: el texto del tweet.
\item
  \textbf{target}: solo en el conjunto de datos de train porqué es la
  variable clase a predecir. Indica si un tweet es sobre un desastre
  real (1) o no (0).
\end{itemize}

\begin{Shaded}
\begin{Highlighting}[]
\KeywordTok{str}\NormalTok{(train, }\DataTypeTok{width =} \DecValTok{85}\NormalTok{, }\DataTypeTok{strict.width =} \StringTok{"cut"}\NormalTok{)}
\end{Highlighting}
\end{Shaded}

\begin{verbatim}
## 'data.frame':    7613 obs. of  5 variables:
##  $ id      : int  1 4 5 6 7 8 10 13 14 15 ...
##  $ keyword : chr  NA NA NA NA ...
##  $ location: chr  NA NA NA NA ...
##  $ text    : chr  "Our Deeds are the Reason of this #earthquake May ALLAH Forgive "..
##  $ target  : int  1 1 1 1 1 1 1 1 1 1 ...
\end{verbatim}

\begin{Shaded}
\begin{Highlighting}[]
\KeywordTok{str}\NormalTok{(test, }\DataTypeTok{width =} \DecValTok{85}\NormalTok{, }\DataTypeTok{strict.width =} \StringTok{"cut"}\NormalTok{)}
\end{Highlighting}
\end{Shaded}

\begin{verbatim}
## 'data.frame':    3263 obs. of  4 variables:
##  $ id      : int  0 2 3 9 11 12 21 22 27 29 ...
##  $ keyword : chr  NA NA NA NA ...
##  $ location: chr  NA NA NA NA ...
##  $ text    : chr  "Just happened a terrible car crash" "Heard about #earthquake is"..
\end{verbatim}

\hypertarget{variable-target}{%
\subsubsection{\texorpdfstring{Variable
\emph{target}}{Variable target}}\label{variable-target}}

Categorizamos la variable a predecir, ya que inicialmente es de tipo
entero, y observamos su distribución.

\begin{Shaded}
\begin{Highlighting}[]
\NormalTok{train}\OperatorTok{$}\NormalTok{target <-}\StringTok{ }\KeywordTok{as.factor}\NormalTok{(}\KeywordTok{ifelse}\NormalTok{(train}\OperatorTok{$}\NormalTok{target }\OperatorTok{==}\StringTok{ }\DecValTok{0}\NormalTok{, }\StringTok{"No"}\NormalTok{, }\StringTok{"Yes"}\NormalTok{))}
\KeywordTok{ggplot}\NormalTok{(train, }\KeywordTok{aes}\NormalTok{(}\DataTypeTok{x=}\NormalTok{target)) }\OperatorTok{+}\StringTok{ }\KeywordTok{geom_bar}\NormalTok{(}\KeywordTok{aes}\NormalTok{(}\DataTypeTok{fill=}\NormalTok{target))}
\end{Highlighting}
\end{Shaded}

\begin{center}\includegraphics[width=0.7\linewidth]{document_files/figure-latex/unnamed-chunk-6-1} \end{center}

La distribución no está muy sesgada y vemos que hay menos tweets que se
refieren a desastres reales. La distribución de la variable a predecir
está relativamente equilibrada, donde el 43\% de las observaciones son
desastrosas y el 57\% no.

\begin{Shaded}
\begin{Highlighting}[]
\KeywordTok{sum}\NormalTok{(train}\OperatorTok{$}\NormalTok{target }\OperatorTok{==}\StringTok{ "Yes"}\NormalTok{) }\OperatorTok{/}\StringTok{ }\KeywordTok{dim}\NormalTok{(train)[}\DecValTok{1}\NormalTok{] }\OperatorTok{*}\StringTok{ }\DecValTok{100}
\end{Highlighting}
\end{Shaded}

\begin{verbatim}
## [1] 42.96598
\end{verbatim}

\begin{Shaded}
\begin{Highlighting}[]
\KeywordTok{sum}\NormalTok{(train}\OperatorTok{$}\NormalTok{target }\OperatorTok{==}\StringTok{ "No"}\NormalTok{) }\OperatorTok{/}\StringTok{ }\KeywordTok{dim}\NormalTok{(train)[}\DecValTok{1}\NormalTok{] }\OperatorTok{*}\StringTok{ }\DecValTok{100}
\end{Highlighting}
\end{Shaded}

\begin{verbatim}
## [1] 57.03402
\end{verbatim}

Tampoco presenta un problema notable de \emph{desbalanceo de clase}
porqué contamos con muchas muestras del caso minoritario.

\hypertarget{variable-keyword}{%
\subsubsection{\texorpdfstring{Variable
\emph{keyword}}{Variable keyword}}\label{variable-keyword}}

La variable \textbf{keyword} representa una palabra representativa de
cada tweet, se muestran las primeras 10.

\begin{Shaded}
\begin{Highlighting}[]
\NormalTok{train }\OperatorTok\StringTok{ }\KeywordTok{select}\NormalTok{(keyword) }\OperatorTok\StringTok{ }\KeywordTok{unique}\NormalTok{() }\OperatorTok\StringTok{ }\KeywordTok{head}\NormalTok{(}\DecValTok{10}\NormalTok{)}
\end{Highlighting}
\end{Shaded}

\begin{verbatim}
##                 keyword
## 1                  <NA>
## 32               ablaze
## 68             accident
## 103          aftershock
## 137 airplane%20accident
## 172           ambulance
## 210         annihilated
## 244        annihilation
## 273          apocalypse
## 305          armageddon
\end{verbatim}

Ahora veremos si la asociación de cada \textbf{keyword} con un
sentimiento indica una relación con la variable a predecir. Para ello
realizaremos un análisis de sentimientos de cada palabra clave.

\emph{El análisis de sentimientos es una técnica de }Machine
Learning\emph{, basada en el
\href{https://www.kdnuggets.com/2017/02/natural-language-processing-key-terms-explained.html}{procesado
del lenguaje natural}, que pretende obtener información subjetiva de una
serie de textos. Su aplicación es este caso, consiste en resolver si un
tweet es real o no real en relación a un desastre.}

Para ello usaremos los paquetes \textbf{syuzhet}, \textbf{ggcorrplot} y
\textbf{doParallel}.

\begin{itemize}
\tightlist
\item
  \textbf{syuzhet} cuenta con la función \textbf{get\_nrc\_sentiment}
  que calcula la presencia de los diferentes sentimientos dado un
  conjunto de textos. Los parámetros de esta función son:

  \begin{itemize}
  \tightlist
  \item
    \textbf{char\_v}. Un vector de caracteres que en este caso contiene
    todas las palabras clave.
  \item
    \textbf{language}. Define el lenguaje.
  \item
    \textbf{cl}. Para análisis paralelo. Es opcional, pero en este caso
    lo usaremos porqué hay muchas palabras clave.
  \end{itemize}
\item
  \textbf{ggcorrplot} muestra una visualización gráfica de una matriz de
  correlación usando \emph{ggplot2}.
\item
  \textbf{doParallel} proporciona una computación paralela. Los
  parámetros de esta función son:

  \begin{itemize}
  \tightlist
  \item
    \textbf{makePSOCKcluster}. Crea un clúster de sockets paralelos.
  \item
    \textbf{registerDoParallel}. Registra el número de \emph{cores} que
    usará el clúster creado.
  \item
    \textbf{stopCluster}. Detiene la computación paralela.
  \end{itemize}
\end{itemize}

Análisis de correlaciones entre \textbf{keyword} y \textbf{target}:

\begin{Shaded}
\begin{Highlighting}[]
\NormalTok{cl <-}\StringTok{ }\KeywordTok{makePSOCKcluster}\NormalTok{(}\DecValTok{4}\NormalTok{, }\DataTypeTok{setup_strategy=}\StringTok{"sequential"}\NormalTok{)}
\KeywordTok{registerDoParallel}\NormalTok{(cl)}

\NormalTok{emocion.df <-}\StringTok{ }\KeywordTok{get_nrc_sentiment}\NormalTok{(}\DataTypeTok{char_v =} \KeywordTok{gsub}\NormalTok{(}\StringTok{"_"}\NormalTok{, }\StringTok{" "}\NormalTok{, train}\OperatorTok{$}\NormalTok{keyword), }
                                \DataTypeTok{language =} \StringTok{"english"}\NormalTok{, }\DataTypeTok{cl=}\NormalTok{cl)}
\NormalTok{emocion.df <-}\StringTok{ }\NormalTok{emocion.df }\OperatorTok\StringTok{ }\KeywordTok{data.frame}\NormalTok{(}\DataTypeTok{target =}\NormalTok{ train}\OperatorTok{$}\NormalTok{target)}
\NormalTok{emocion.df}\OperatorTok{$}\NormalTok{target <-}\StringTok{ }\KeywordTok{as.numeric}\NormalTok{(emocion.df}\OperatorTok{$}\NormalTok{target)}

\KeywordTok{cor}\NormalTok{(emocion.df) }\OperatorTok\StringTok{ }
\StringTok{  }\KeywordTok{ggcorrplot}\NormalTok{(}\DataTypeTok{lab =} \OtherTok{TRUE}\NormalTok{, }
             \DataTypeTok{title =} \StringTok{"Matriz de correlación entre los }\CharTok{\textbackslash{}n}\StringTok{sentimientos de keyword y target"}\NormalTok{,}
             \DataTypeTok{legend.title =} \StringTok{"correlation"}\NormalTok{)}
\end{Highlighting}
\end{Shaded}

\begin{center}\includegraphics[width=0.7\linewidth]{document_files/figure-latex/unnamed-chunk-9-1} \end{center}

\begin{Shaded}
\begin{Highlighting}[]
\KeywordTok{stopCluster}\NormalTok{(cl)}
\end{Highlighting}
\end{Shaded}

Al observar la matriz de correlaciones, se observa una correlación nula
con cada uno de los sentimientos. Esto hace que esta variable
explicativa no sea buena para hacer una predicción.

\hypertarget{variable-location}{%
\subsubsection{\texorpdfstring{Variable
\emph{location}}{Variable location}}\label{variable-location}}

La variable \textbf{location} representa la ubicación desde donde se
generaron los tweets, se muestran las primeras 10.

\begin{Shaded}
\begin{Highlighting}[]
\NormalTok{train }\OperatorTok\StringTok{ }\KeywordTok{select}\NormalTok{(location) }\OperatorTok\StringTok{ }\KeywordTok{unique}\NormalTok{() }\OperatorTok\StringTok{ }\KeywordTok{head}\NormalTok{(}\DecValTok{10}\NormalTok{)}
\end{Highlighting}
\end{Shaded}

\begin{verbatim}
##                         location
## 1                           <NA>
## 32                    Birmingham
## 33 Est. September 2012 - Bristol
## 34                        AFRICA
## 35              Philadelphia, PA
## 36                    London, UK
## 37                      Pretoria
## 38                  World Wide!!
## 40                Paranaque City
## 41                Live On Webcam
\end{verbatim}

\begin{Shaded}
\begin{Highlighting}[]
\KeywordTok{count}\NormalTok{(train }\OperatorTok\StringTok{ }\KeywordTok{select}\NormalTok{(location) }\OperatorTok\StringTok{ }\KeywordTok{unique}\NormalTok{())}
\end{Highlighting}
\end{Shaded}

\begin{verbatim}
##      n
## 1 3342
\end{verbatim}

En total hay 3342 ubicaciones. A continuación, mostramos las ubicaciones
más frecuentes:

\begin{Shaded}
\begin{Highlighting}[]
\NormalTok{location.freq <-}\StringTok{ }\KeywordTok{table}\NormalTok{(}\KeywordTok{unlist}\NormalTok{(train }\OperatorTok\StringTok{ }\KeywordTok{select}\NormalTok{(location)))}
\NormalTok{location.freq[}\KeywordTok{which}\NormalTok{(location.freq }\OperatorTok{>}\StringTok{ }\DecValTok{10}\NormalTok{)]}
\end{Highlighting}
\end{Shaded}

\begin{verbatim}
## 
##         Australia        California   California, USA            Canada 
##                18                17                15                29 
##           Chicago       Chicago, IL             Earth        Everywhere 
##                11                18                11                15 
##           Florida             India         Indonesia           Ireland 
##                14                24                13                12 
##             Kenya            London       Los Angeles   Los Angeles, CA 
##                20                45                13                26 
##            Mumbai          New York      New York, NY           Nigeria 
##                22                71                15                28 
##               NYC     San Francisco San Francisco, CA           Seattle 
##                12                14                11                11 
##           Toronto                UK    United Kingdom     United States 
##                12                27                14                50 
##               USA  Washington, D.C.    Washington, DC         Worldwide 
##               104                13                21                19
\end{verbatim}

\begin{Shaded}
\begin{Highlighting}[]
\KeywordTok{barplot}\NormalTok{(location.freq[}\KeywordTok{which}\NormalTok{(location.freq}\OperatorTok{>}\DecValTok{10}\NormalTok{)], }\DataTypeTok{las =} \DecValTok{2}\NormalTok{,  }
        \DataTypeTok{ylab =} \StringTok{"Frequency"}\NormalTok{)}
\end{Highlighting}
\end{Shaded}

\begin{center}\includegraphics[width=0.7\linewidth]{document_files/figure-latex/unnamed-chunk-11-1} \end{center}

Del total de ubicaciones (3342), la mayoría de ellas cuenta con menos de
10 observaciones. Esto hace que esta variable explicativa tampoco sea
buena para hacer una predicción.

\hypertarget{variable-text}{%
\subsubsection{\texorpdfstring{Variable
\emph{text}}{Variable text}}\label{variable-text}}

Hemos considerado que las variables explicativas \textbf{keyword} y
\textbf{location} no son buenas para hacer una predicción, así que nos
centraremos en la variable \textbf{text}.

Llegados a este punto unimos los conjuntos de train y test (\emph{7613 +
3263 observaciones}) para poder extraer los sentimientos más adelante.

\begin{Shaded}
\begin{Highlighting}[]
\NormalTok{complete_df <-}\StringTok{ }\KeywordTok{bind_rows}\NormalTok{(train, test)}
\KeywordTok{dim}\NormalTok{(complete_df)}
\end{Highlighting}
\end{Shaded}

\begin{verbatim}
## [1] 10876     5
\end{verbatim}

Echamos un vistazo más de cerca a las variables del nuevo conjunto de
datos \textbf{complete\_df}.

\begin{Shaded}
\begin{Highlighting}[]
\KeywordTok{summary}\NormalTok{(complete_df)}
\end{Highlighting}
\end{Shaded}

\begin{verbatim}
##        id          keyword            location             text          
##  Min.   :    0   Length:10876       Length:10876       Length:10876      
##  1st Qu.: 2719   Class :character   Class :character   Class :character  
##  Median : 5438   Mode  :character   Mode  :character   Mode  :character  
##  Mean   : 5438                                                           
##  3rd Qu.: 8156                                                           
##  Max.   :10875                                                           
##   target    
##  No  :4342  
##  Yes :3271  
##  NA's:3263  
##             
##             
## 
\end{verbatim}

La variable \textbf{id} es solo un identificador único y la
eliminaremos.

\begin{Shaded}
\begin{Highlighting}[]
\NormalTok{complete_df}\OperatorTok{$}\NormalTok{id <-}\StringTok{ }\OtherTok{NULL}
\end{Highlighting}
\end{Shaded}

Observamos si existen valores perdidos.

\begin{Shaded}
\begin{Highlighting}[]
\KeywordTok{colSums}\NormalTok{(}\KeywordTok{sapply}\NormalTok{(complete_df, is.na))}
\end{Highlighting}
\end{Shaded}

\begin{verbatim}
##  keyword location     text   target 
##       87     3638        0     3263
\end{verbatim}

Las variables explicativas \textbf{keyword} y \textbf{location}
contienen valores perdidos. Sobretodo hay una gran cantidad de tweets,
para los cuales falta su ubicación. No existen valores perdidos para la
variable explicativa \textbf{text}, tampoco para la variable a predecir
\textbf{target}. Los 3263 valores perdidos de la variable a predecir
provienen del conjunto de datos de test. Nos ocuparemos de los valores
perdidos más adelante.

Parece que la variable explicativa \textbf{text} es una buena elección
para una buena predicción y basaremos los siguientes pasos en ella.

\hypertarget{procesamiento-de-texto}{%
\subsection{Procesamiento de texto}\label{procesamiento-de-texto}}

Como en todo procesamiento de lenguaje natural, realizaremos el
procesamiento de un conjunto de textos. En este caso realizaremos un
procesamiento de los textos de los tweets y los prepararemos para el
modelado. Comencemos por crear un corpus de los mensajes de texto de los
tweets. Para ello usaremos la función \textbf{Corpus} del paquete
\textbf{tm}, que creará nuestro corpus a partir de un vector de textos.
La función \textbf{VectorSource} interpretará cada mensaje de texto de
los tweets como un elemento del vector de textos.

\emph{Un corpus lingüístico se define como ``un conjunto de textos de un
mismo origen'' y que tiene por función recopilar un conjunto de textos.
El uso de un corpus lingüístico nos permitirá obtener información de las
palabras utilizadas con más o menor frecuencia.}

\begin{Shaded}
\begin{Highlighting}[]
\NormalTok{myCorpus <-}\StringTok{ }\KeywordTok{Corpus}\NormalTok{(}\KeywordTok{VectorSource}\NormalTok{(complete_df}\OperatorTok{$}\NormalTok{text))}
\end{Highlighting}
\end{Shaded}

Durante el procesamiento de texto seguiremos la transformación de un
mensaje de tweet específico para ver como se modifica a medida que
avanzamos en el procesamiento de texto. Este mensaje es:

\begin{Shaded}
\begin{Highlighting}[]
\KeywordTok{paste0}\NormalTok{(myCorpus[[}\DecValTok{400}\NormalTok{]])}
\end{Highlighting}
\end{Shaded}

\begin{verbatim}
## [1] "Jewish leaders prayed at the hospital where a Palestinian family is being treated after arson http://t.co/Wf8iTK2KVx via @HuffPostRelig"
\end{verbatim}

Dividimos el procesamiento de texto en 7 pasos.

\begin{enumerate}
\def\labelenumi{\arabic{enumi}.}
\tightlist
\item
  Eliminar enlaces.
\end{enumerate}

\begin{Shaded}
\begin{Highlighting}[]
\NormalTok{removeURL <-}\StringTok{ }\ControlFlowTok{function}\NormalTok{(x) }\KeywordTok{gsub}\NormalTok{(}\StringTok{"http[^[:space:]]*"}\NormalTok{, }\StringTok{""}\NormalTok{, x)  }
\NormalTok{myCorpus <-}\StringTok{ }\KeywordTok{tm_map}\NormalTok{(myCorpus, }\KeywordTok{content_transformer}\NormalTok{(removeURL))}
\KeywordTok{paste0}\NormalTok{(myCorpus[[}\DecValTok{400}\NormalTok{]])}
\end{Highlighting}
\end{Shaded}

\begin{verbatim}
## [1] "Jewish leaders prayed at the hospital where a Palestinian family is being treated after arson  via @HuffPostRelig"
\end{verbatim}

Hemos eliminado: \emph{\url{http://t.co/Wf8iTK2KVx}}.

La función \textbf{gsub} busca y reemplaza desde la primera hasta todas
las coincidencias de un patrón (que normalmente representa una
\emph{regular expression}). La función \textbf{tm\_map} es la encargada
de aplicar las diferentes transformaciones de los textos al corpus
creado.

\begin{enumerate}
\def\labelenumi{\arabic{enumi}.}
\setcounter{enumi}{1}
\tightlist
\item
  Convertir a minúsculas.
\end{enumerate}

\begin{Shaded}
\begin{Highlighting}[]
\NormalTok{myCorpus <-}\StringTok{ }\KeywordTok{tm_map}\NormalTok{(myCorpus, }\KeywordTok{content_transformer}\NormalTok{(stri_trans_tolower))}
\KeywordTok{paste0}\NormalTok{(myCorpus[[}\DecValTok{400}\NormalTok{]])}
\end{Highlighting}
\end{Shaded}

\begin{verbatim}
## [1] "jewish leaders prayed at the hospital where a palestinian family is being treated after arson  via @huffpostrelig"
\end{verbatim}

\begin{enumerate}
\def\labelenumi{\arabic{enumi}.}
\setcounter{enumi}{2}
\tightlist
\item
  Eliminar los nombres de usuario.
\end{enumerate}

\begin{Shaded}
\begin{Highlighting}[]
\NormalTok{removeUsername <-}\StringTok{ }\ControlFlowTok{function}\NormalTok{(x) }\KeywordTok{gsub}\NormalTok{(}\StringTok{"@[^[:space:]]*"}\NormalTok{, }\StringTok{""}\NormalTok{, x)  }
\NormalTok{myCorpus <-}\StringTok{ }\KeywordTok{tm_map}\NormalTok{(myCorpus, }\KeywordTok{content_transformer}\NormalTok{(removeUsername))}
\KeywordTok{paste0}\NormalTok{(myCorpus[[}\DecValTok{400}\NormalTok{]])}
\end{Highlighting}
\end{Shaded}

\begin{verbatim}
## [1] "jewish leaders prayed at the hospital where a palestinian family is being treated after arson  via "
\end{verbatim}

Hemos eliminado: \emph{@huffpostrelig}.

\begin{enumerate}
\def\labelenumi{\arabic{enumi}.}
\setcounter{enumi}{3}
\tightlist
\item
  Eliminar todo excepto el idioma y el espacio en inglés.
\end{enumerate}

\begin{Shaded}
\begin{Highlighting}[]
\NormalTok{removeNumPunct <-}\StringTok{ }\ControlFlowTok{function}\NormalTok{(x) }\KeywordTok{gsub}\NormalTok{(}\StringTok{"[^[:alpha:][:space:]]*"}\NormalTok{, }\StringTok{""}\NormalTok{, x)   }
\NormalTok{myCorpus <-}\StringTok{ }\KeywordTok{tm_map}\NormalTok{(myCorpus, }\KeywordTok{content_transformer}\NormalTok{(removeNumPunct))}
\KeywordTok{paste0}\NormalTok{(myCorpus[[}\DecValTok{400}\NormalTok{]])}
\end{Highlighting}
\end{Shaded}

\begin{verbatim}
## [1] "jewish leaders prayed at the hospital where a palestinian family is being treated after arson  via "
\end{verbatim}

No se observan cambios en en ejemplo.

\begin{enumerate}
\def\labelenumi{\arabic{enumi}.}
\setcounter{enumi}{4}
\tightlist
\item
  Eliminar palabras irrelevantes (eliminación de redundancias).
\end{enumerate}

\begin{Shaded}
\begin{Highlighting}[]
\NormalTok{myStopWords <-}\StringTok{ }\KeywordTok{c}\NormalTok{((}\KeywordTok{stopwords}\NormalTok{(}\StringTok{'english'}\NormalTok{)), }
           \KeywordTok{c}\NormalTok{(}\StringTok{"really"}\NormalTok{, }\StringTok{"tweets"}\NormalTok{, }\StringTok{"saw"}\NormalTok{, }\StringTok{"just"}\NormalTok{, }\StringTok{"feel"}\NormalTok{, }\StringTok{"may"}\NormalTok{, }\StringTok{"us"}\NormalTok{, }\StringTok{"rt"}\NormalTok{, }\StringTok{"every"}\NormalTok{, }\StringTok{"one"}\NormalTok{,}
             \StringTok{"amp"}\NormalTok{, }\StringTok{"like"}\NormalTok{, }\StringTok{"will"}\NormalTok{, }\StringTok{"got"}\NormalTok{, }\StringTok{"new"}\NormalTok{, }\StringTok{"can"}\NormalTok{, }\StringTok{"still"}\NormalTok{, }\StringTok{"back"}\NormalTok{, }\StringTok{"top"}\NormalTok{, }\StringTok{"much"}\NormalTok{,}
             \StringTok{"near"}\NormalTok{, }\StringTok{"im"}\NormalTok{, }\StringTok{"see"}\NormalTok{, }\StringTok{"via"}\NormalTok{, }\StringTok{"get"}\NormalTok{, }\StringTok{"now"}\NormalTok{, }\StringTok{"come"}\NormalTok{, }\StringTok{"oil"}\NormalTok{, }\StringTok{"let"}\NormalTok{, }\StringTok{"god"}\NormalTok{, }\StringTok{"want"}\NormalTok{,}
             \StringTok{"pm"}\NormalTok{, }\StringTok{"last"}\NormalTok{, }\StringTok{"hope"}\NormalTok{, }\StringTok{"since"}\NormalTok{, }\StringTok{"everyone"}\NormalTok{, }\StringTok{"food"}\NormalTok{, }\StringTok{"content"}\NormalTok{, }\StringTok{"always"}\NormalTok{, }\StringTok{"th"}\NormalTok{,}
             \StringTok{"full"}\NormalTok{, }\StringTok{"found"}\NormalTok{, }\StringTok{"dont"}\NormalTok{, }\StringTok{"look"}\NormalTok{, }\StringTok{"cant"}\NormalTok{, }\StringTok{"mh"}\NormalTok{, }\StringTok{"lol"}\NormalTok{, }\StringTok{"set"}\NormalTok{, }\StringTok{"old"}\NormalTok{, }\StringTok{"service"}\NormalTok{,}
             \StringTok{"city"}\NormalTok{, }\StringTok{"home"}\NormalTok{, }\StringTok{"live"}\NormalTok{, }\StringTok{"night"}\NormalTok{, }\StringTok{"news"}\NormalTok{, }\StringTok{"say"}\NormalTok{, }\StringTok{"video"}\NormalTok{, }\StringTok{"people"}\NormalTok{, }\StringTok{"ill"}\NormalTok{, }
             \StringTok{"way"}\NormalTok{,  }\StringTok{"please"}\NormalTok{, }\StringTok{"years"}\NormalTok{, }\StringTok{"take"}\NormalTok{, }\StringTok{"homes"}\NormalTok{, }\StringTok{"read"}\NormalTok{, }\StringTok{"man"}\NormalTok{, }\StringTok{"next"}\NormalTok{, }\StringTok{"cross"}\NormalTok{, }
             \StringTok{"boy"}\NormalTok{, }\StringTok{"bad"}\NormalTok{, }\StringTok{"ass"}\NormalTok{))}

\NormalTok{myCorpus <-}\StringTok{ }\KeywordTok{tm_map}\NormalTok{(myCorpus, removeWords, myStopWords) }
\KeywordTok{paste0}\NormalTok{(myCorpus[[}\DecValTok{400}\NormalTok{]])}
\end{Highlighting}
\end{Shaded}

\begin{verbatim}
## [1] "jewish leaders prayed   hospital   palestinian family   treated  arson   "
\end{verbatim}

Hemos eliminado: at the where a is being after via.

Las palabras irrelevantes que hemos eliminado se denominan \emph{stop
words o palabras vacías}. Cada idioma tiene sus propias palabras vacías.
Como los textos están en inglés hemos eliminado los \emph{stop words}
que pertenecen al inglés usando la función \textbf{stopwords}, además
hemos añadido aleatóriamente alguna de las palabras vacías más usadas en
los mensajes de texto de los tweets (ver
\url{https://techland.time.com/2009/06/08/the-500-most-frequently-used-words-on-twitter/}).

\emph{Las stop words o palabras vacías son todas aquellas palabras que
carecen de un significado por si solas. Suelen ser artículos,
preposiciones, conjunciones, pronombres, etc.}

\begin{enumerate}
\def\labelenumi{\arabic{enumi}.}
\setcounter{enumi}{5}
\tightlist
\item
  Eliminar palabras de una sola letra.
\end{enumerate}

\begin{Shaded}
\begin{Highlighting}[]
\NormalTok{removeSingle <-}\StringTok{ }\ControlFlowTok{function}\NormalTok{(x) }\KeywordTok{gsub}\NormalTok{(}\StringTok{" . "}\NormalTok{, }\StringTok{" "}\NormalTok{, x)   }
\NormalTok{myCorpus <-}\StringTok{ }\KeywordTok{tm_map}\NormalTok{(myCorpus, }\KeywordTok{content_transformer}\NormalTok{(removeSingle))}
\KeywordTok{paste0}\NormalTok{(myCorpus[[}\DecValTok{400}\NormalTok{]])}
\end{Highlighting}
\end{Shaded}

\begin{verbatim}
## [1] "jewish leaders prayed hospital palestinian family treated  arson "
\end{verbatim}

No se observan cambios en en ejemplo.

\begin{enumerate}
\def\labelenumi{\arabic{enumi}.}
\setcounter{enumi}{6}
\tightlist
\item
  Eliminar espacios en blanco adicionales.
\end{enumerate}

\begin{Shaded}
\begin{Highlighting}[]
\NormalTok{myCorpus <-}\StringTok{ }\KeywordTok{tm_map}\NormalTok{(myCorpus, stripWhitespace)}
\KeywordTok{paste0}\NormalTok{(myCorpus[[}\DecValTok{400}\NormalTok{]])}
\end{Highlighting}
\end{Shaded}

\begin{verbatim}
## [1] "jewish leaders prayed hospital palestinian family treated arson "
\end{verbatim}

Terminamos con el procesamiento de texto. A continuación, crearemos dos
\emph{Term Document Matrix} (matriz que describe la frecuencia de las
palabras que se producen en una colección de textos) para un análisis de
sentimientos más detallado. Usaremos la función
\textbf{TermDocumentMatrix} y dividiremos el corpus en dos, según el
número de elementos de los conjuntos de datos train y test. Recordamos
que el conjunto de datos de train contiene 7613 observaciones, y el
conjunto de datos de test contiene 3263 observaciones. El parámetro
\textbf{control} evalúa cada texto de la matriz, concretamente se
evaluaran todas las palabras de cada texto (no aplicamos ningún filtro).

\begin{Shaded}
\begin{Highlighting}[]
\NormalTok{train_tdm <-}\StringTok{ }\KeywordTok{TermDocumentMatrix}\NormalTok{(myCorpus[}\DecValTok{1}\OperatorTok{:}\DecValTok{7613}\NormalTok{], }
                                \DataTypeTok{control=} \KeywordTok{list}\NormalTok{(}\DataTypeTok{wordLengths=} \KeywordTok{c}\NormalTok{(}\DecValTok{1}\NormalTok{, }\OtherTok{Inf}\NormalTok{)))}
\NormalTok{test_tdm <-}\StringTok{ }\KeywordTok{TermDocumentMatrix}\NormalTok{(myCorpus[}\DecValTok{7614}\OperatorTok{:}\DecValTok{10876}\NormalTok{], }
                               \DataTypeTok{control=} \KeywordTok{list}\NormalTok{(}\DataTypeTok{wordLengths=} \KeywordTok{c}\NormalTok{(}\DecValTok{1}\NormalTok{, }\OtherTok{Inf}\NormalTok{)))}
\NormalTok{train_tdm}
\end{Highlighting}
\end{Shaded}

\begin{verbatim}
## <<TermDocumentMatrix (terms: 14825, documents: 7613)>>
## Non-/sparse entries: 58707/112804018
## Sparsity           : 100%
## Maximal term length: 49
## Weighting          : term frequency (tf)
\end{verbatim}

\begin{Shaded}
\begin{Highlighting}[]
\NormalTok{test_tdm}
\end{Highlighting}
\end{Shaded}

\begin{verbatim}
## <<TermDocumentMatrix (terms: 8966, documents: 3263)>>
## Non-/sparse entries: 25434/29230624
## Sparsity           : 100%
## Maximal term length: 35
## Weighting          : term frequency (tf)
\end{verbatim}

\end{document}
