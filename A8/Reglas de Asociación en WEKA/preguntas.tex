\documentclass[11pt]{exam}
\usepackage[utf8]{inputenc}
\usepackage{hyperref}
\usepackage{graphicx}

\title{Reglas de Asociación en Weka}
\author{Laura Rodríguez Navas \\ rodrigueznavas@posgrado.uimp.es}
\date{\today}

\pagestyle{plain}

\begin{document}

\maketitle

En esta práctica se realiza un estudio acerca de los datos del hundimiento del Titanic a través de la herramienta \href{https://www.cs.waikato.ac.nz/ml/weka/}{Weka}. Los datos se encuentran en la dirección \url{http://www.hakank.org/weka/titanic.arff} y corresponden a las características de los 2201 pasajeros del Titanic. Estos datos son reales y se han obtenido de \textit{"Report on the Loss of the 'Titanic' (S.S.)" (1990), British Board of Trade Inquiry Report\_(reprint), Gloucester, UK: Allan Sutton Publishing}.

\vspace{3mm}

Para realizar esta práctica, se debe cargar el dataset Titanic que se ha descargado anteriormente y contestar a las siguientes preguntas:

\begin{questions}

% Pregunta 1
{\question Cuando ejecutamos el algoritmo Apriori de Weka, podemos utilizar diferentes umbrales de soporte. Dependiendo de qué umbrales de soporte pongamos, nos saldrán más o menos itemsets. Como resultado, Weka nos proporciona un conjunto de ítems L(1)... L(4) cuyos números van variando conforme cambiamos el umbral de soporte. 
	
Responde a las siguientes preguntas, utilizando capturas de pantalla y explicando los resultados de manera clara y concisa:}

\begin{parts}
\part ¿Qué representan cada uno de estos conjuntos de ítems? 

\renewcommand{\figurename}{Figura}

\begin{figure}[h]
	\centering
	\includegraphics[scale=0.5]{Captura_1_1.png}
	\caption{soporte = 0.15 y confianza = 0.9.}
	\label{Captura_1_1}
\end{figure}

\begin{enumerate}
	\item El conjunto de ítems L(1) representa el número de conjuntos de ítems de tamaño 1 encontrados en el dataset. Que en este caso son 7.
	\item El conjunto de ítems L(2) representa el número de conjuntos de ítems de tamaño 2 encontrados en el dataset. Que en este caso son 13.
	\item El conjunto de ítems L(3) representa el número de conjuntos de ítems de tamaño 3 encontrados en el dataset. Que en este caso son 8.
	\item El conjunto de ítems L(4) representa el número de conjuntos de ítems de tamaño 4 encontrados en el dataset. Que en este caso son 2.
\end{enumerate}	

\part ¿Puede existir L(0)? Explica porqué.

No puede existir L(0). El conjunto de ítems L(0) representa el número de conjuntos de ítems de tamaño 0, es decir, el número de conjuntos de ítems vacíos, y el conjunto vacío ($\emptyset$) no es válido como conjunto de ítems.

\part ¿Puede existir L(5)? Explica porqué.

No puede existir L(5) porqué el dataset de Titanic solo contiene cuatro atributos diferentes. (ver Figura \ref{Captura_1_2}). 

\begin{figure}[h]
	\begin{center}
		\includegraphics[scale=0.3]{Captura_1_2.png}
	\end{center}
\caption{características del dataset.}
\label{Captura_1_2}
\end{figure}

\part ¿Puede L(1) tomar un valor mayor que 10? Explica de manera teórica que eso no es posible y compruébalo experimentalmente.

L(1) no puede tomar un valor mayor que 10. El número de conjuntos de ítems de tamaño 1 será 9. Ya que, como máximo, el número de conjuntos de ítems de tamaño 1, cuenta con los diferentes valores de cada atributo del dataset. 

\newpage
En este caso, el dataset de Titanic, contiene 9 valores diferentes:

\begin{itemize}
	\item Class ("1st", "2nd", "3rd", "Crew")
	\item Age {"Adult", "Child"}
	\item Sex {"Male", "Female"}
	\item Survived {"Yes", "No"}
\end{itemize}

Para comprobarlo experimentalmente, el umbral de confianza debe ser igual a 1. (ver Figura \ref{Captura_1_3})

\begin{figure}[h]
	\centering
	\includegraphics[scale=0.5]{Captura_1_3.png}
	\caption{soporte = 0.1 y confianza = 1.}
	\label{Captura_1_3}
\end{figure}

\end{parts}

% Pregunta 2
{\question Además de los valores de soporte, el algoritmo Apriori de Weka nos permite utilizar diferentes umbrales de soporte y confianza. Responde a las siguientes preguntas, utilizando capturas de pantalla y explicando los resultados de manera clara y concisa:}

\begin{parts}
\part ¿Es posible que una regla tenga un valor de soporte inferior a su confianza? Explica porqué y demuéstralo experimentalmente.

Es posible. Para explicar el porqué y demostrarlo nos ayudaremos de la métrica \textit{lift} (mejora de la confianza). La métrica lift mide la relación del valor del soporte observado y del valor del soporte esperado de los conjuntos de ítems que forman las reglas, si esos conjuntos fueran independientes. Los conjuntos de ítems estarán positivamente correlacionados si el valor de lift es superior a 1, si el valor es inferior a 1 estarán negativamente correlacionados.

Si una regla puede tener un valor de soporte inferior a su confianza, el valor de lift será superior a uno, y nos indicará que el conjunto de ítems que forman la regla aparecen una cantidad de veces superior a lo esperado, por lo que se puede intuir que la regla hace que los conjuntos de ítems que la forman aparezcan más de lo normal. Por ejemplo, la regla: 

\vspace{2mm}
\textit{class=crew 885 ==$>$ age=adult 885    $<$conf:(1)$>$ lift:(1.05) lev:(0.02) [43] conv:(43.83)}
\vspace{2mm}

\newpage
Vemos que el número de instancias donde aparecen los conjuntos de ítems es elevado (885) y el valor de lift es superior a uno. Eso nos permite saber que existen muchas ocurrencias entre estos dos conjuntos y que la regla será potencialmente útil para predecir en futuros conjuntos de datos. Ha resultado ser la regla número uno (Ver Figura \ref{Captura_1_4}).

\begin{figure}[h]
	\centering
	\includegraphics[scale=0.45]{Captura_1_4.png}
	\caption{soporte inferior a confianza.}
	\label{Captura_1_4}
\end{figure}

\part ¿Es posible que una regla tenga un valor de confianza inferior a su suporte? Explica porqué y demuéstralo experimentalmente.

También es posible. Una regla puede tener un valor de confianza inferior a su soporte. Aunque en este caso el valor de lift será inferior a uno, y nos indicará que el conjunto de ítems que forman la regla aparecen una cantidad de veces inferior a lo esperado,por lo que se puede intuir que la regla hace que los conjuntos de ítems que la forman aparezcan menos de lo normal. Por ejemplo, la regla: 

\vspace{2mm}
\textit{class=3rd 706 ==$>$ age=adult sex=male 462    $<$conf:(0.65)$>$ lift:(0.86) lev:(-0.03) [-72] conv:(0.7)}
\vspace{2mm}

\begin{figure}[h]
	\centering
	\includegraphics[scale=0.45]{Captura_1_5.png}
	\caption{soporte superior a confianza.}
	\label{Captura_1_5}
\end{figure}

Vemos que el número de instancias donde aparecen los conjuntos de datos no son tan elevados (706 y 462), además no lo hacen de la misma forma, y el valor de lift es inferior a uno. Eso nos permite saber que existen menos ocurrencias entre los conjuntos de ítems que forman la regla y que esto tiene un efecto negativo entre ellos. 

Potencialmente no será tan útil para predecir en futuros conjuntos de datos. Ha resultado ser la regla número uno (Ver Figura \ref{Captura_1_5}).

\part La variación del umbral de confianza (dado un umbral fijo de soporte) no afecta a los conjuntos L(1)... L(4). ¿Por qué?
\end{parts}

Porqué la variación del umbral de confianza es una métrica enfocada para las reglas, y mide la frecuencia con que se pueden encontrar. En cambio, el umbral de soporte es una métrica enfocada para los conjuntos de ítems, que mide la proporción de estos dentro del dataset. Y como los  conjuntos L(1)... L(4) representan el número de apariciones de conjuntos de ítems según su tamaño dentro del dataset, el umbral les afectará.

% Pregunta 3
{\question Usaremos ahora, 0.75 como valor mínimo de soporte y de confianza 0.00. Comprobamos que obtenemos dos reglas de asociación, sin embargo, L(2) es 1. ¿Qué quiere decir esto? ¿A qué corresponde L(2)? ¿Qué itemset representa?}

Ver Figura \ref{Captura_1_6}.

\begin{figure}[h]
	\centering
	\includegraphics[scale=0.5]{Captura_1_6.png}
	\caption{Soporte = 0.75 y confianza = 0.}
	\label{Captura_1_6}
\end{figure}

\newpage
% Pregunta 4
{\question Analiza el conjunto de reglas que salen al aplicar diferentes umbrales de soporte y confianza. Coge una regla, la que veas más interesante, y coméntala. Explica sus valores de métricas y qué representan, y el significado de la regla, es decir, el conocimiento que te aporta dicha regla.}

Después de aplicar diferentes umbrales de soporte y confianza la regla elegida es la siguiente:

\begin{figure}[h]
	\centering
	\includegraphics[scale=0.5]{Captura_1_7.png}
	\caption{tripulación que no sobrevivió.}
	\label{Captura_1_7}
\end{figure}

Esta regla nos indica que la mayoría de personas de la tripulación que no sobrevivieron al naufragio del Titanic fueron personas adultas. Las métricas que podemos observar en ella son:

\begin{itemize}
	\item La confianza con valor igual a 1. Lo que significa que para el 100\% de las transacciones del dataset que contienen los conjuntos de ítems que forman la regla, la regla se cumplirá siempre. Es decir, siempre que una persona de la tripulación no haya sobrevivido, esta persona será adulta.
	\item La mejora de la confianza o lift, con valor igual a 1.05. Como se ha comentado anteriormente, si el valor de lift es superior a 1, permite saber que hay muchas ocurrencias entre los conjuntos de ítems que forman la regla y que dependen unas de ellas. Eso hace que la regla sea potencialmente útil para predecir en futuros conjuntos de datos.
	\item La influencia o \textit{leverage}, con valor igual a 0.02. Es una métrica muy parecida a lift, ya que mide la 
	\item La convicción o \textit{conviction}, con valor igual a 33.33. Esta métrica nos indica el grado de implicación de la regla dentro del dataset. Hay cierto grado de implicación, la regla puede señalar un aspecto concluyente del dataset. Es decir, podríamos afirmar que muchas de las personas que no sobrevivieron fueron personas adultas de la tripulación, exactamente un 33.33\% del total de pasajeros a bordo.
\end{itemize}

\end{questions}

\end{document}
