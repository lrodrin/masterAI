\documentclass{uimppracticas}

%Permitir cabeceras y pie de páginas personalizados
\pagestyle{fancy}

%Path por defecto de las imágenes
\graphicspath{ {./images/} }

%Declarar formato de encabezado y pie de página de las páginas del documento
\fancypagestyle{doc}{
  %Pie de Página
  \footerpr{}{}{{\thepage} de \pageref{LastPage}}
}

%Declarar formato de encabezado y pie del título e indice
\fancypagestyle{titu}{%
  %Cabecera
  \headerpr{}{}{}
  %Pie de Página
  \footerpr{}{}{}
}

\appto\frontmatter{\pagestyle{titu}}
\appto\mainmatter{\pagestyle{doc}}

\begin{document}
	
%Comienzo formato título
\frontmatter

%Portada (Centrado todo)
\centeredtitle{./images/LogoUIMP.png}{Máster Universitario en Investigación en Inteligencia Artificial}{Curso 2020-2021}{Recuperación y extracción de información, \\ grafos y redes sociales}{Análisis y Visualización Básica de una Red Social con Gephi}

\begin{center}
\large \today
\end{center}

\vspace{40mm}

\begin{flushright}
 	{\bf Laura Rodríguez Navas}\\
 	\textbf{DNI:} 43630508Z\\
 	\textbf{e-mail:} \href{rodrigueznavas@posgrado.uimp.es}{rodrigueznavas@posgrado.uimp.es}
\end{flushright}

\newpage

%Índice
% \tableofcontents

% \newpage

%Comienzo formato documento general
\mainmatter

\setlength\parskip{2.5ex}

\section*{La Red}

La red \textit{Diseasome}\cite{Goh8685} seleccionada para realizar esta práctica, es una red no dirigida de trastornos y genes de diferentes enfermedades vinculadas por asociaciones conocidas entre trastornos y genes, que indican el origen genético común de muchas enfermedades. La red está formada por 526 enfermedades y 903 genes, donde los genes asociados con trastornos similares muestran una mayor probabilidad de interacciones físicas entre sus diagnósticos y una mayor similitud de perfiles de expresión para sus tratamientos, lo que respalda la existencia de distintos clústers funcionales específicos de cada enfermedad. 

El conjunto de datos de \textit{Diseasome} viene como un archivo \textit{.zip}, que se puede descargar \href{http://gephi.org/datasets/diseasome.gexf.zip}{aquí}. Que una vez se ha descargado y descomprimido, obtenemos un archivo \textit{.gexf}, un archivo de grafos. Importamos el archivo de grafos a \textit{Gephi}\cite{Gephi} y empezamos a probar diferentes opciones de visualización.

Después de probar diferentes visualizaciones nos decidimos por el algoritmo de distribución: \href{https://github.com/gephi/gephi/wiki/Fruchterman-Reingold}{Fruchterman Reingold}\cite{fruchterman1991graph} (en la ventana \textit{Distribución}). Para evitar que las componentes conexas queden fuera de la vista principal, fijamos el valor del parámetro \textit{Gravedad} a 20 y también marcamos las opciones \textit{Disuadir Hubs y/o Evitar el solapamiento}. Esto convertirá nuestra visualización de la red en un círculo y colocará la red alrededor de una misma área (ver Figura \ref{completa_negro_FR}). 

\begin{figure}[H]
	\centering
	\includegraphics[width=0.6\textwidth]{images/completa_negro_FR}
	\caption{Red completa sobre un fondo negro sin etiquetas.}
	\label{completa_negro_FR}
\end{figure}

De una primera visualización pasamos a la detección de comunidades para colorear los clústers de la red. \textit{Gephi} implementa el método de \href{http://perso.uclouvain.be/vincent.blondel/research/louvain.html}{Louvain}\cite{Blondel2008} para la detección de comunidades (disponible en el panel de \textit{Estadísticas}). Para ello, damos clic en ejecutar \textit{Modularidad} y veremos como el algoritmo de detección de comunidades nos ha creado un nuevo parámetro de particionamiento llamado \textit{Modularity Class}. Si seleccionamos este nuevo parámetro podremos observar las comunidades encontradas y si finalmente pulsamos \textit{Aplicar} colorearemos los nodos según las comunidades encontradas. Esto hace que la visualización sea más colorida y se vea bien donde se encuentra cada comunidad. También añadimos etiquetas a los nodos (ver Figura \ref{completa_FR_labels}). 

\begin{figure}[H]
	\centering
	\includegraphics[width=0.85\textwidth]{images/completa_FR_labels}
	\caption{Red completa sobre un fondo blanco con etiquetas.}
	\label{completa_FR_labels}
\end{figure}

Como podemos observar en la Figura \ref{completa_FR_labels}, las diferentes comunidades están agrupadas por colores. El tamaño de las etiquetas depende del tamaño del nodo. Está claro que los cánceres son la enfermedad más dominante de todas, siendo una de las enfermedades más comunes en comparación con otras enfermedades que existen en la actualidad. Un dato curioso es que la sordera es la enfermedad que se lleva la mayor porción. También vemos otros clústers además de los cánceres, como la diabetes, la salud mental, etc. Se da la propiedad libre de escala (\textit{scale-free}), muy común en redes reales, porque muchos nodos de la red poseen un gran número de enlaces a otros nodos.

\section*{Análisis Básico de la Red}

Como pudimos observar en la sección anterior, la red parece demasiado compleja para analizarla visualmente, así que para los primeros pasos del análisis de la red, comenzamos por anotar los valores de las medidas globales básicas: el número de nodos ($N$) y el número de enlaces ($L$), que aparecen directamente en la ventana \textit{Contexto}. El número de nodos de la red es igual a 1419 y el número de enlaces es igual a 3926. Además calculamos manualmente el número máximo de enlaces $L_{max}$.

\begin{center}
	$L_{max} = \frac{N\,*\,(N-1)}{2} = \frac{1419\,*\,(1419-1)}{2} = 1006071$
\end{center}

Posteriormente calculamos otra medida global, el grado medio <k>, ejecutando la opción correspondiente en la ventana \textit{Estadísticas}. El valor del grado medio <k> es igual a 5,533, es decir, que cada trastorno de la red está conectado con 5 genes en media. Al realizar el cálculo del grado medio <k>, también obtenemos la distribución de grados de la red completa (ver Figura \ref{degree-distribution}).

\begin{figure}[H]
	\centering
	\includegraphics[width=0.55\textwidth]{images/degree-distribution}
	\caption{Distribución de grados de la red completa.}
	\label{degree-distribution}
\end{figure}

Existen algunas enfermedades fuertemente conectadas (\textit{hubs}), la mayor con grado 135. Concretamente 10 enfermedades tienen más de 50 variaciones de genes.

La opción \textit{Densidad} de grafo mide la relación entre el número de enlaces ($L$) y el número máximo de enlaces ($L_{max}$). Ejecutamos la opción y vemos que su valor es igual a 0,004 (valor cercano a 0, densidad mínima). Esto nos indica que el grafo es un grafo disperso, el número de enlaces ($L$) no es cercano al número de máximo de enlaces ($L_{max}$). 

A continuación, ejecutamos la opción \textit{Coeficiente medio de clustering <C>} para obtener la medida del mismo nombre. El valor del coeficiente medio de clustering es igual a 0,819. Es un valor muy alto que nos indica un grado muy significativo de clustering local. Al realizar el cálculo del coeficiente medio de clustering también obtenemos la distribución de coeficientes de clustering de la red completa (ver Figura \ref{clustering-coefficient}), donde vemos que el coeficiente de clustering es mucho mayor en los nodos poco conectados que en los nodos más conectados (\textit{hubs}). En este caso, los nodos de grado bajo se sitúan en vecindarios localmente densos y viceversa como consecuencia de la jerarquía de la red.

\begin{figure}[H]
	\centering
	\includegraphics[width=0.55\textwidth]{images/clustering-coefficient}
	\caption{Distribución de coeficientes de clustering de la red completa.}
	\label{clustering-coefficient}
\end{figure}

Ahora, pasamos a analizar la conectividad de la red. En primer lugar, obtenemos el número de componentes conexas ejecutando la opción \textit{Componentes conexos} (disponible en el panel de \textit{Estadísticas}). Vemos  que el número de componentes conexas es igual a 1. En este caso, como solo tenemos una componente conexa, determinamos que la componente gigante de la red es la red completa actual.

Finalmente, calculamos las medidas globales restantes (diámetro $d_{max}$ y distancia media <d>) ejecutando la opción correspondiente al \textit{Diámetro de la red} en la ventana {Estadísticas}. El valor del diámetro ($d_{max}$) es igual a 15. Viendo la red (ver Figura \ref{completa_FR_labels}), pensaríamos que hay variaciones grandes en las distancias entre los nodos pero la red tiene una distancia media baja (<d> = 6,783). El cálculo del diámetro también nos proporciona los valores de las tres medidas de centralidad (intermediación, cercanía y excentricidad), que podemos observar en las figuras \ref{Betweenness-Centrality-Distribution}, \ref{Closeness-Centrality-Distribution} y \ref{Eccentricity-Distribution}. En la Figura \ref{Betweenness-Centrality-Distribution} podemos comprobar que no hay variaciones grandes en las distancias entre los nodos, la mayoría de los nodos están bastante juntos. Solo 4 se encuentran muy alejados del resto.

\begin{figure}[H]
	\centering
	\includegraphics[width=0.55\textwidth]{images/Betweenness-Centrality-Distribution}
	\caption{Centralidad de intermediación no normalizada de la red completa.}
	\label{Betweenness-Centrality-Distribution}
\end{figure}

\begin{figure}[H]
	\centering
	\includegraphics[width=0.55\textwidth]{images/Closeness-Centrality-Distribution}
	\caption{Centralidad de cercanía no normalizada de la red completa.}
	\label{Closeness-Centrality-Distribution}
\end{figure}

Observando la Figura \ref{Closeness-Centrality-Distribution} parece que se de la propiedad de mundos pequeños (\textit{small-world}). La mayoría de los nodos no son vecinos entre sí, y sin embargo la mayoría de los nodos pueden ser alcanzados desde cualquier nodo origen a través de un número relativamente corto de saltos entre ellos. También observamos que no existen distancias largas en la red y que bastantes nodos no están muy alejados del centro de la red.

\begin{figure}[H]
	\centering
	\includegraphics[width=0.55\textwidth]{images/Eccentricity-Distribution}
	\caption{Centralidad de excentricidad no normalizada de la red completa.}
	\label{Eccentricity-Distribution}
\end{figure}

Observando la Figura \ref{Eccentricity-Distribution} vemos que la excentricidad no es muy constante. Esto nos indica que al no ser constante, no existen ni nodos muy periféricos ni muy centrales en la red.

A continuación, mostramos la tabla que resume los valores de las medidas calculadas anteriormente (ver fichero \textit{MedidasRedesPracticaParteI-1.xlsx} para observar la tabla en formato Excel).

\begin{table}[H]
	\centering
	\begin{tabular}{|l|r|}
		\hline
		\textbf{Medida}                                            & \textbf{Valor} \\ \hline\hline
		Número de nodos N                                          & 1419           \\ \hline
		Número de enlaces L                                        & 3926           \\ \hline
		Número máximo de enlaces Lmax                              & 1006071        \\ \hline
		Densidad del grafo L/Lmax                                  & 0.004          \\ \hline
		Grado medio \textless{}k\textgreater{}                     & 5.533          \\ \hline
		Diámetro dmax                                              & 15             \\ \hline
		Distancia media d                                          & 6.783          \\ \hline
		Coeficiente medio de clustering \textless{}C\textgreater{} & 0.819          \\ \hline
		Número de componentes conexas                              & 1              \\ \hline
		Número de nodos componente gigante                         & 1419           \\ \hline
		Número de aristas componente gigante                       & 3926           \\ \hline
	\end{tabular}
	\caption{Tabla con los valores de las medidas estudiadas.}
	\label{tabla}
\end{table}

En la siguiente sección de la práctica empleamos la medidas de centralidad calculadas.

\newpage

\section*{Estudio de la Centralidad de los Actores}

En esta sección se realiza un pequeño análisis de redes sociales sobre nuestra red basado en las medidas de centralidad. El análisis determina los 5 actores principales de la red mediante las medidas de centralidad de grado, intermediación, cercanía y vector propio.

Los valores de tres de estas medidas (grado, intermediación y cercanía) ya están calculados en los pasos que se han realizado en la sección anterior. La centralidad de grado (no normalizada) se generó al calcular el \textit{Grado medio <k>} en la ventana \textit{Estadísticas}. Las medidas de centralidad de intermediación y cercanía (no normalizadas) se generaron con la opción \textit{Diámetro de la red}. En este caso, las volvemos a calcular para obtener las medidas normalizadas con el checkbox \textit{Normalizar centralidades en el rango [0,1]}.

\begin{figure}[H]
	\centering
	\includegraphics[width=0.6\textwidth]{images/Betweenness-Centrality-Distribution-Norm}
	\caption{Centralidad de intermediación normalizada de la red completa.}
	\label{Betweenness-Centrality-Distribution-Norm}
\end{figure}

\begin{figure}[H]
	\centering
	\includegraphics[width=0.6\textwidth]{images/Closeness-Centrality-Distribution-Norm}
	\caption{Centralidad de cercanía normalizada de la red completa.}
	\label{Closeness-Centrality-Distribution-Norm}
\end{figure}

Finalmente, calculamos la centralidad de vector propio que se calcula en la opción del menú \textit{Estadísticas} del mismo nombre (ver Figura \ref{Eigenvector-Centrality}).

\begin{figure}[H]
	\centering
	\includegraphics[width=0.6\textwidth]{images/Eigenvector-Centrality}
	\caption{Centralidad de vector propio de la red completa.}
	\label{Eigenvector-Centrality}
\end{figure}

Una vez ejecutadas las opciones de menú correspondientes, los valores de centralidad de cada nodo pueden visualizarse en la tabla \textit{Nodos} de la pestaña \textit{Tabla de datos}, junto con el resto de la información asociada a cada nodo. A continuación, anotamos los nombres de los 5 actores con mejor valor para cada una de las cuatro medidas anteriores, así como el valor de la medida en cada caso y los almacenamos en la tabla siguiente:

\begin{table}[H]
	\centering
	\begin{tabular}{ |c|c|c|c| } 
		\hline
		\thead{\textbf{Centralidad de} \\ \textbf{Grado}} & \thead{\textbf{Centralidad de} \\ \textbf{Intermediación}} & 	\thead{\textbf{Centralidad de } \\ \textbf{Cercanía}} & \thead{\textbf{Centralidad de } \\ \textbf{Vector propio}} \\ 
		\hline
		\thead{Colon Cancer \\ 134} & \thead{Cardiomyopathy \\ 0.445069}  & \thead{Lipodystrophy \\ 0.245414} & \thead{Colon cancer \\ 1.000000} \\ 
		\hline
		\thead{Deafness \\ 91} & \thead{Lipodystrophy \\ 0.400534} & \thead{Diabetes mellitus \\ 0.244272} & \thead{Breast cancer \\ 0.731462} \\ 
		\hline
		\thead{Leukemia \\ 89} & \thead{Diabetes mellitus \\ 0.28795} & \thead{Glioblastoma \\ 0.240402} & \thead{Thyroid carcinoma \\ 0.577893} \\ 
		\hline
		\thead{Breast Cancer \\ 79} & \thead{Glioblastoma \\ 0.234041} & \thead{Obesity \\ 0.228084} & \thead{Pancreatic cancer \\ 0.557891} \\ 
		\hline
		\thead{Diabetes mellitus \\ 75} & \thead{Deafness \\ 0.19061} & \thead{Cardiomyopathy \\ 0.226771} & \thead{Gastric cancer \\ 0.479706} \\ 
		\hline 
	\end{tabular}
	\caption{Los 5 actores con mejor valor por medida de centralidad.}
	\label{tabla2}
\end{table}

Los actores mas importantes de la red desde una perspectiva global en función de los valores de las medidas de centralidad pertenecen a cánceres, diabetes, sordera, cardiopatías, VIH y obesidad (ver Tabla \ref{tabla2}). 

De estos actores realizamos un pequeño análisis a continuación, sin tener en cuenta la centralidad de grado, que aunque refleja el número de conexiones de cada actor, no tiene en cuenta la estructura global de la red.

Como sabemos, una medida bastante importante es la centralidad de intermediación, que nos indica que actores hacen de puente entre otras regiones de la red. Por lo cual pueden conectar distintas comunidades entre sí. En el caso que nos ocupa (\textit{Diseasome}), podemos observar en la Tabla \ref{tabla2} que las cardiopatías son el actor mayoritario de esta medida de centralidad. Eso es que las cardiopatías son la enfermedad más conectada a otras enfermedades distintas, podemos decir que se diagnostica junto a muchísimas otras enfermedades. Por ejemplo, podemos sufrir cardipatías con la obesidad, con la diabetes, etc. En la Figura \ref{Betweenness-Centrality} se puede observar como el nodo correspondiente a las cardiopatías tiene la mayor intensidad de color de la red.

La centralidad de cercanía mide cómo de cerca está un actor del centro de la red. En este caso parece ser que el actor más centrado pertenece a la enfermedad lipodistrofia, que se asocia a enfermedades como el virus de la inmunodeficiencia humana (VIH). Pero está medida no nos serviría de mucho en un análisis más profundo, ya que muchos de los actores tienen un valor parecido de centralidad de cercanía. En la tabla \ref{tabla2} no se puede apreciar bien esta conclusión, pero si observamos la Figura \ref{Closeness-Centrality} o la Figura \ref{closnesscentrality}, se puede observar perfectamente. En la Figura \ref{Closeness-Centrality} vemos diferentes nodos con la misma intensidad de color, y en la Figura \ref{closnesscentrality} muchas enfermedades con valores parecidos de centralidad de cercanía.

Por último, la medida de centralidad de vector propio es una medida recursiva que asigna importancia a un nodo en función de la importancia de sus vecinos. Es decir, tiene en cuenta la calidad de las conexiones, en lugar de la cantidad. El primer actor, el cáncer de colon, tiene un valor de esta medida de 1, lo cual indica que es el nodo más importante y con el mayor número de conexiones importantes. Es el actor a tener más en cuenta de la red. Tanto en la Figura \ref{Eigenvector-Centrality-Graph} y Figura \ref{eigencentrality} se puede observar.

\newpage

\section*{Visualizaciones y Gráficos adicionales}

\begin{figure}[H]
	\centering
	\includegraphics[width=0.75\textwidth]{images/Grado-Centrality}
	\caption{Centralidad de grado de la red completa.}
	\label{Grado-Centrality}
\end{figure}

\begin{figure}[H]
	\centering
	\includegraphics[width=0.75\textwidth]{images/Betweenness-Centrality}
	\caption{Centralidad de intermediación de la red completa.}
	\label{Betweenness-Centrality}
\end{figure}

\begin{figure}[H]
	\centering
	\includegraphics[width=0.75\textwidth]{images/Closeness-Centrality}
	\caption{Centralidad de cercanía de la red completa.}
	\label{Closeness-Centrality}
\end{figure}

\begin{figure}[H]
	\centering
	\includegraphics[width=0.75\textwidth]{images/Eigenvector-Centrality-Graph}
	\caption{Centralidad de vector propio de la red completa.}
	\label{Eigenvector-Centrality-Graph}
\end{figure}

\begin{figure}[H]
	\centering
	\includegraphics[width=0.65\textwidth]{images/degree}
	\caption{Actores de la medida de centralidad de grado.}
	\label{degree}
\end{figure}

\begin{figure}[H]
	\centering
	\includegraphics[width=0.65\textwidth]{images/betweenesscentrality}
	\caption{Actores de la medida de centralidad de intermediación.}
	\label{betweenesscentrality}
\end{figure}

\begin{figure}[H]
	\centering
	\includegraphics[width=0.65\textwidth]{images/closnesscentrality}
	\caption{Actores de la medida de centralidad de cercanía.}
	\label{closnesscentrality}
\end{figure}

\begin{figure}[H]
	\centering
	\includegraphics[width=0.65\textwidth]{images/eigencentrality}
	\caption{Actores de la medida de centralidad de vector propio.}
	\label{eigencentrality}
\end{figure}

Para realizar los gráficos anteriores, que representan los valores de las medidas para todos los actores de las enfermedades de la red en ejes de coordenadas, donde se han excluido algunas enfermedades y todos los genes, para poder mejorar las visualizaciones, se ha utilizado un \textit{script} en Python que se incluye en los ficheros de la práctica y que también podemos ver a continuación:

\begin{lstlisting}[language=Python, basicstyle=\small]
import seaborn as sns
import matplotlib.pyplot as plt
import pandas as pd


def createGraph(values, col):
	plt.figure(num=None, figsize=(20, 18), dpi=80, facecolor='w', edgecolor='r')
	g = sns.barplot(x="Label", y=col, data=values)
	plt.xticks(rotation=90)
	plt.savefig('images/{}.png'.format(col))


if __name__ == '__main__':
	df = pd.read_csv("dataset/diseasome.csv", sep=",")
	df_disease = df[df['0'].str.contains("disease")]
	df_disease = df_disease[df_disease["degree"] > 15]
	
	colNames = ["degree", "betweenesscentrality", "closnesscentrality", "eigencentrality"]
	for colName in colNames:
		createGraph(df_disease, colName)
\end{lstlisting}

\newpage

\renewcommand{\refname}{Bibliografía}
\bibliographystyle{unsrt}
\bibliography{biblio}

\end{document}
