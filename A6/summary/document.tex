\documentclass[a4paper]{article}
\usepackage[utf8]{inputenc}
\title{Description-Oriented Community Detection using Exhaustive Subgroup Discovery}

\date{Summarised: DATE}
\author{Martin Atzmuellera, Stephan Doerfela, Folke Mitzlaffa}
\begin{document}
\maketitle
\section{Justification}
Communities can intuitively be defined as subsets of nodes of a graph with a dense structure in the corresponding subgraph. However, for mining such communities usually only structural aspects are taken into account. Typically, no concise nor easily interpretable community description is provided. 

For tackling this issue, this paper focuses on description-oriented community detection using subgroup discovery. In order to provide both structurally valid and interpretable communities we utilize the graph structure as well as additional descriptive features of the graph’s nodes.A descriptive community pattern built upon these features then describes and identifies a community.

We aim at identifying communities according to standard community quality measures, while providing characteristic descriptions of these communities at the same time. For this task, we propose several optimistic estimates of standard community quality functions to be used for efficient pruning of the search space in an exhaustive branch-and-bound algorithm.

//
Quality measure?
Graph space?
Branch-and-bound algorithm?

idea principal - we approach the task of identifying communities as sets of nodes together with a description. Such a community pattern then provides an intuitive description of the community. his is usually not achieved by classical community mining methods that consider the nodes of a network (e. g., denoting users in a social network) as mere strings or ids.

Per què la investigació es duu a terme?
Because the while classic community detection, e. g., [17] for a survey, just identifies sub- groups of nodes with a dense structure, lacking an interpretable description, this paper focuses on the task of description-oriented comunity detection.
%\begin{itemize}
  %\item 
  %\item 
  %\item 
%\end{itemize}

\section{Main Points}
\begin{itemize}
  \item 
  \item 
\end{itemize}

\section{Results}
\begin{itemize}
  \item
\end{itemize}
\end{document}