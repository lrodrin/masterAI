\documentclass[a4paper]{article}
\usepackage[utf8]{inputenc}
\title{Description-Oriented Community Detection using Exhaustive Subgroup Discovery}

\date{Summarised: January 9, 2020}
\author{Martin Atzmuellera, Stephan Doerfela, Folke Mitzlaffa}
\begin{document}
\maketitle
\section{Abstract}
For mining of communities, defining communities as subsets of nodes of a graph with a dense structure in the corresponding subgraph, usually only the structural aspects of the communities are taken into account. Typically, no concise nor easily interpretable for each community description is provided.

This paper focuses on description-oriented community detection using subgroup discovery to provide both structurally valid and interpretable communities, with the graph structure and descriptive features of the graph’s nodes. A descriptive community pattern, created with the descriptive features, describes and identifies each community, i. e., a set of nodes, and vice versa.
Essentially, patterns are looked up characterizing interesting set of nodes (i. e., subgroups) in the graph. The interestingness set of nodes that forms a community is evaluated by a selectable standard community quality measure.
The selection is based on several optimistic estimates of standard quality functions used for efficient pruning of the search space in an exhaustive branch and bound algorithm.

Finally, the approach is evaluated using five real world datasets, obtained from different social media applications.

\section{Main Points}

\section{Results}

\end{document}