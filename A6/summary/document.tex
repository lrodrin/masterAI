\documentclass[a4paper]{article}
\usepackage[utf8]{inputenc}
\title{Description-Oriented Community Detection using Exhaustive Subgroup Discovery}

\date{Summarised: January 9, 2020}
\author{Martin Atzmuellera, Stephan Doerfela, Folke Mitzlaffa}
\begin{document}
\maketitle
\section{Abstract}
Usually, for mining of communities, defining communities as subsets of nodes of a graph with dense structure in the corresponding sub graph, only the structural aspects of this communities are taken into account. Typically, for each community, their description provided is no concise and easily interpretable. 
\vskip 0.3cm

This article is focuses on the description-oriented community detection, using subgroup discovery to provide an structurally valid and interpretable description for the communities, with a graph's structures and descriptive features of a graph’s nodes. A descriptive community pattern, created with descriptive features of a graph, describes and identifies each community and vice versa. Essentially, patterns are looked up characterizing interesting set of nodes (i. e., subgroups) in the graph. The interestingness set of nodes that forms a community is evaluated by a selectable standard community's quality measure. The selection is based on several optimistic estimates of standard quality functions used for efficient pruning of the search space in an exhaustive branch and bound algorithm. 
\vskip 0.3cm

The approach of this article is evaluated using five real world datasets, obtained from different social media applications.

\section{Introduction}
Since 2010, the classical community detection in graphs, just identify subgroups of nodes of a graph with dense structure, lacking an interpretable description.
\vskip 0.3cm

As a innovation, this article focuses on the task of description-oriented community detection, based on additional descriptive functions of the nodes of a graph contained in the network, to identify communities such as sets of nodes, along with a description. Specifically, for the identification is used a logical formula on the values of nodes, named community pattern, that describes characteristics of the nodes, providing and intuitive description of the community by and easily interpretable conjunction of attribute-value pairs.
\vskip 0.3cm

The approach of this article is usually not achieved by classical community mining methods, that consider the nodes of a network as mere strings or ids.
\vskip 0.3cm

The great interest of this article is that the discovered method can solve problems that are generally not addressed through standard approaches to community detection. Also because the community patterns can easily be incorporated into practical applications, for example, for recommendations in social bookmarking systems, like BibSonomy or delicious.
\vskip 0.3cm

In contrast to global approaches, the method is exceptional. Because given community quality measures to local partitioning of a network , potentially overlapping communities can be considered. Also, is not limited to any systems and can be applied to any kind of graph-structured data for which additional descriptive features are available.

\section{Main Points}
The contribution of this paper is structured as follows:
\vskip 0.3cm

First, is introduced the approach for description-driven community detection using exhaustive subgroup discovery and is mentioned the COMODO algorithm for the search of community patterns through a quality measure. The COMODO algorithm is a branch and bound algorithm, that use an efficient branch and bound method with appropriate pruning techniques based on exhaustive subgroup discovery using optimistic estimates. The implemented COMODO algorithm contains a pruning scheme that makes the approach scalable for longer data sets.
\vskip 0.3cm
	
The proposed optimistic estimates for a range of standard community quality functions, which are efficient to compute faster the description-oriented community detection using the COMODO algorithm. Specifically, are consider a number of standard community quality functions: The segregation index, the inverse average ODF (out degree fraction) and the local modularity. In addition, the different community quality measures are discussed and extended the optimistic estimates accordingly. 
\vskip 0.3cm

In the following X, summarizes basics of subgroup discovery, and provides general notions of graphs and community quality measures. 
\vskip 0.3cm

Next, introduce the proposed approach for description-oriented community detection and presents a number of optimistic estimates for standard community evaluation functions. 
\vskip 0.3cm

After that, discusses related work. 
\vskip 0.3cm

For demonstrating the efficiency of proposed optimistic estimates and validity of the obtained community patterns, are performed experiments using five different data sets. The experiments demonstrated the effectiveness of proposed descriptive mining approach applying the presented optimistic estimates. Furthermore, the implemented is concerns the validity of the patterns by the structural properties of the patterns and the sub graphs induced by the respective community patterns, and compared COMODO to three baseline community detection algorithms.
 Then discusses their results in the context of three real-world applications. The results indicate statisdically valid and significant results that de not exhibit the typical problems and pathological cases such as small communities sizes that are often encountered when using typical community mining methods. Because the COMODO lagorithm is able to detect communities that are typically captured by shorter descriptions leading to a lower description complexity, compared to the baseline.
\vskip 0.3cm

Finally, concludes the paper with a summary and directions for future work. Apply method in more diverse data and extend the approach for community detection on dynamic networks.
% emplenar amb conclusions

\end{document}