\documentclass[a4paper]{article}
\usepackage[utf8]{inputenc}
\title{Description-Oriented Community Detection using Exhaustive Subgroup Discovery}

\date{Summarised: January 9, 2020}
\author{Martin Atzmuellera, Stephan Doerfela, Folke Mitzlaffa}
\begin{document}
\maketitle
\section{Abstract}
For mining of communities, defining communities as subsets of nodes of a graph with a dense structure in the corresponding subgraph, usually only the structural aspects of the communities are taken into account. Typically, no concise nor easily interpretable for each community description is provided.

This paper focuses on description-oriented community detection using subgroup discovery to provide both structurally valid and interpretable communities, with the graph structure and descriptive features of the graph’s nodes. A descriptive community pattern, created with the descriptive features, describes and identifies each community, i. e., a set of nodes, and vice versa.
Essentially, patterns are looked up characterizing interesting set of nodes (i. e., subgroups) in the graph. The interestingness set of nodes that forms a community is evaluated by a selectable standard community quality measure.
The selection is based on several optimistic estimates of standard quality functions used for efficient pruning of the search space in an exhaustive branch and bound algorithm.

Finally, the approach is evaluated using five real world datasets, obtained from different social media applications.

\section{Introduction}
Since 2010, the classical community detection in graphs for a survey, just identifies subgroups of nodes with dense structure, lacking an interpretable description.
As a novelty, this work focuses on the task of description-oriented community detection based on additional descriptive functions of the nodes contained in the network, to identify communities such as sets of nodes, along with a description. Specifically, with a logical formula on the values of nodes, named community pattern, that describes characteristics of the nodes, providing and intuitive description of the community by and easily interpretable conjunction of attribute-value pairs.
This approach is usually not achieved by classical community mining methods that consider the nodes of a network as mere strings or ids.

The great interest for this article is that the method is discovered can resolve problems that typically are not addressed by standard approaches for community detection. And the community patterns can easily be incorporated into a practical application, for example, for recommendations in social bookmarking systems.
In contrast to global approaches, the method is exceptional. Because given community quality measures to local partitioning of a network can be considered, potentially overlapping communities. Also, is not limited to such systems and can be applied to any kind of graph-structured data for which additional descriptive features are available.

\section{Main Points}

\end{document}