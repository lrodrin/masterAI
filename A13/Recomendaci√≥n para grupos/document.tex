\documentclass{uimppracticas}

%Permitir cabeceras y pie de páginas personalizados
\pagestyle{fancy}

%Path por defecto de las imágenes
\graphicspath{ {./images/} }

%Declarar formato de encabezado y pie de página de las páginas del documento
\fancypagestyle{doc}{
  %Pie de Página
  \footerpr{}{}{{\thepage} de \pageref{LastPage}}
}

%Declarar formato de encabezado y pie del título e indice
\fancypagestyle{titu}{%
  %Cabecera
  \headerpr{}{}{}
  %Pie de Página
  \footerpr{}{}{}
}

\appto\frontmatter{\pagestyle{titu}}
\appto\mainmatter{\pagestyle{doc}}

\begin{document}
	
%Comienzo formato título
\frontmatter

%Portada (Centrado todo)
\centeredtitle{./images/LogoUIMP.png}{Máster Universitario en Investigación en Inteligencia Artificial}{Curso 2020-2021}{Sistemas de Recomendación}{Recomendación para grupos \\ en Python}

\begin{center}
\large \today
\end{center}

\vspace{40mm}

\begin{flushright}
 	{\bf Laura Rodríguez Navas}\\
 	\textbf{DNI:} 43630508Z\\
 	\textbf{e-mail:} \href{rodrigueznavas@posgrado.uimp.es}{rodrigueznavas@posgrado.uimp.es}
\end{flushright}

\newpage

%Índice
\tableofcontents

\newpage

%Comienzo formato documento general
\mainmatter

\setlength\parskip{2.5ex}

\section{Introducción}

\section{Conjunto de datos}

El conjunto de datos que se ha usado en esta práctica se encuentra disponible públicamente para su descarga en el siguiente enlace: \url{https://www.kaggle.com/abhikjha/movielens-100k/download}. Este conjunto de datos llamado \textit{ml-latest-small}, describe las calificaciones (entre 1 y 5 estrellas) y la actividad del etiquetado de \href{http://movielens.org}{MovieLens}~\cite{MovieLens}, un servicio de recomendación de películas. Concretamente, el conjunto de datos contiene 100836 clasificaciones y 3683 etiquetas de 9742 películas. Los datos fueron creados por 610 usuarios que fueron seleccionados al azar. Cada usuario clasificó al menos 20 películas y está representado por una identificación.

Una vez descargado el conjunto de datos veremos que está contenido en los archivos \textit{links.csv}, \textit{movies.csv}, \textit{ratings.csv} y \textit{tags.csv}. Pero para el desarrollo de esta práctica solo se utilizan los archivos \textit{movies.csv} y \textit{ratings.csv}. A continuación cargamos cada archivo dentro de su dataframe (ver Definición~\ref{dataframe}) con el uso de la librería pandas~\cite{pandas}.

\begin{lstlisting}[language=python]
movies_df = pd.read_csv('dataset/movies.csv')
ratings_df = pd.read_csv('dataset/ratings.csv')
\end{lstlisting}

Observamos el contenido de \textit{movies\_df} para ver como ha quedado organizado:

\begin{table}[h]
	\centering
	\begin{tabular}{rll}
		\toprule
		movieId &                               title &                                       genres \\
		\midrule
		1 &                    Toy Story (1995) &  Adventure|Animation|Children|Comedy|Fantasy \\
		2 &                      Jumanji (1995) &                   Adventure|Children|Fantasy \\
		3 &             Grumpier Old Men (1995) &                               Comedy|Romance \\
		4 &            Waiting to Exhale (1995) &                         Comedy|Drama|Romance \\
		5 &  Father of the Bride Part II (1995) &                                       Comedy \\
		\bottomrule
	\end{tabular}
	\caption{Contenido del dataframe \textit{movies\_df} inicial.}
	\label{movies_df}
\end{table}

Cada película tiene un único identificador, un título con su año de estreno y diferentes géneros. Como los años contienen caracteres \textit{unicode} y para que no haya problemas más adelante, los sacaremos de la columna de los títulos y los ubicaremos en su propia columna que nombraremos \textit{year}. Para ello, primero utilizamos una expresión regular para encontrar los años guardados entre paréntesis, y con la función \textit{extract} de la librería pandas los extraemos de la columna de los títulos para guardarlos en su propia columna. Después borramos los años de la columna de los títulos y con la función \textit{strip} nos aseguramos de sacar los espacios finales que pudieran haber.

\begin{lstlisting}[language=python]
regular_expression = r'\((.*?)\)'
movies_df['year'] = movies_df.title.str.lower().str.extract(regular_expression)
movies_df['title'] = movies_df.title.str.replace(regular_expression, '', regex=True)
movies_df['title'] = movies_df['title'].apply(lambda x: x.strip())
\end{lstlisting}

\newpage

Vemos el resultado:

\begin{table}[h]
	\centering
	\begin{tabular}{rlll}
		\toprule
		movieId &                        title &                                       genres &  year \\
		\midrule
		1 &                    Toy Story &  Adventure|Animation|Children|Comedy|Fantasy &  1995 \\
		2 &                      Jumanji &                   Adventure|Children|Fantasy &  1995 \\
		3 &             Grumpier Old Men &                               Comedy|Romance &  1995 \\
		4 &            Waiting to Exhale &                         Comedy|Drama|Romance &  1995 \\
		5 &  Father of the Bride Part II &                                       Comedy &  1995 \\
		\bottomrule
	\end{tabular}
\caption{Contenido del dataframe \textit{movies\_df} con la columna \textit{year}.}
\label{movies_df_years}
\end{table}

\begin{definition}\label{dataframe}
Un DataFrame es una estructura de datos etiquetada bidimensional que acepta diferentes tipos datos de entrada organizados en columnas. Se puede pensar en un DataFrame como una hoja de cálculo o una tabla SQL.
\end{definition}

Sacamos la columna de los géneros, ya que no los usaremos para este sistema de recomendaciones.

\begin{lstlisting}[language=python]
movies_df = movies_df.drop('genres', 1)
\end{lstlisting}

Y así queda el dataframe \textit{movies\_df} final:

\begin{table}[h]
	\centering
	\begin{tabular}{rll}
		\toprule
		movieId &                        title &  year \\
		\midrule
		1 &                    Toy Story &  1995 \\
		2 &                      Jumanji &  1995 \\
		3 &             Grumpier Old Men &  1995 \\
		4 &            Waiting to Exhale &  1995 \\
		5 &  Father of the Bride Part II &  1995 \\
		\bottomrule
	\end{tabular}
	\caption{Contenido del dataframe \textit{movies\_df} final.}
	\label{movies_df_final}
\end{table}

\renewcommand{\refname}{Bibliografía}
\bibliographystyle{unsrt}
\bibliography{biblio}

\end{document}
