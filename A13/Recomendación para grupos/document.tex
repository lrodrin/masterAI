\documentclass{uimppracticas}

%Permitir cabeceras y pie de páginas personalizados
\pagestyle{fancy}

%Path por defecto de las imágenes
\graphicspath{ {./images/} }

%Declarar formato de encabezado y pie de página de las páginas del documento
\fancypagestyle{doc}{
  %Pie de Página
  \footerpr{}{}{{\thepage} de \pageref{LastPage}}
}

%Declarar formato de encabezado y pie del título e indice
\fancypagestyle{titu}{%
  %Cabecera
  \headerpr{}{}{}
  %Pie de Página
  \footerpr{}{}{}
}

\appto\frontmatter{\pagestyle{titu}}
\appto\mainmatter{\pagestyle{doc}}

\begin{document}
	
%Comienzo formato título
\frontmatter

%Portada (Centrado todo)
\centeredtitle{./images/LogoUIMP.png}{Máster Universitario en Investigación en Inteligencia Artificial}{Curso 2020-2021}{Sistemas de Recomendación}{Recomendación para grupos \\ en Python}

\begin{center}
\large \today
\end{center}

\vspace{40mm}

\begin{flushright}
 	{\bf Laura Rodríguez Navas}\\
 	\textbf{DNI:} 43630508Z\\
 	\textbf{e-mail:} \href{rodrigueznavas@posgrado.uimp.es}{rodrigueznavas@posgrado.uimp.es}
\end{flushright}

\newpage

%Índice
\tableofcontents

\newpage

%Comienzo formato documento general
\mainmatter

\setlength\parskip{2.5ex}

\section{Introducción}

\section{Conjunto de datos}

Para descargar el conjunto de datos: \href{https://www.kaggle.com/abhikjha/movielens-100k/download}{MovieLens 100k}.

Crearemos dos dataframes utilizando la librería pandas~\cite{jeff_reback_2020_4309786}, a partir de los archivos descargados: \textit{movies.csv} y \textit{ratings.csv}.

\begin{lstlisting}[language=python]
movies_df = pd.read_csv('dataset/movies.csv')
ratings_df = pd.read_csv('dataset/ratings.csv')
\end{lstlisting}

Observamos el contenido de \textit{movies\_df} a continuación:

\begin{table}[h]
	\centering
	\begin{tabular}{rll}
		\toprule
		movieId &                               title &                                       genres \\
		\midrule
		1 &                    Toy Story (1995) &  Adventure|Animation|Children|Comedy|Fantasy \\
		2 &                      Jumanji (1995) &                   Adventure|Children|Fantasy \\
		3 &             Grumpier Old Men (1995) &                               Comedy|Romance \\
		4 &            Waiting to Exhale (1995) &                         Comedy|Drama|Romance \\
		5 &  Father of the Bride Part II (1995) &                                       Comedy \\
		\bottomrule
	\end{tabular}
	\caption{Contenido del dataframe \textit{movies\_df}.}
	\label{movies_df}
\end{table}

\begin{definition}\label{dataframe}
Un DataFrame es una estructura de datos etiquetada bidimensional que acepta diferentes tipos datos de entrada organizados en columnas. Se puede pensar en un DataFrame como una hoja de cálculo o una tabla SQL.
\end{definition}

\newpage
\renewcommand{\refname}{Bibliografía}
\bibliographystyle{unsrt}
\bibliography{biblio}

\end{document}
